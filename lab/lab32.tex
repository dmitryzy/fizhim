\section{Лабораторная работа 6 }
\textbf{Тема:}Определение коэффициента электропроводности сильного электролита.

\textbf{Цель работы:} Изучение влияния разбавления на электропроводность сильных электролитов. Определение коэффициента электропроводности сильного электролита методом измерения электропроводности раствора. Графическое построение зависимостей удельной электропроводности от концентрации и эквивалентной электропроводимости от разбавления. 


\textbf{Оборудование и реактивы:} кондуктометр, мерный цилиндр на 100 мл, мерные колбы на 100 мл, раствор $KCl$, раствор $NaCl$, стакан, вода.

\textbf{Теория}
\textit{\textbf{1. Электропроводность сильных электролитов }}
В водных растворах сильные электролиты находятся преимущественно в  виде  ионов.  При разбавлении эквивалентная проводимость растворов увеличивается,  достигая предельного  значения $\lambda_{0}$. Удельная электропроводность тоже растет, но кривая зависимости ее от разбавления проходит через максимум.  Влияние разбавления на электропроводимость растворов сильных электролитов объясняется ослаблением межионного  взаимодействия  при уменьшении  концентрации электролита. Электрическое взаимодействие ионов влияет на характер их распределения  в  растворах,  особенно   в   концентрированных. Вследствии этого  число  ионов в растворе становится как бы меньше числа, соответствующего их концентрации при полной диссоциации,  в связи с  этим  вводится понятие коэффициента электропроводности $f$:
$$f=\frac{\lambda}{\lambda_{0}}$$
При разбавлении величина f растет, приближаясь к единице. Таким образом,  коэффициент  электропроводности f учитывает взаимодействие между ионами, т.е. показывает во сколько раз действительное значение  электропроводности  меньше теоретического. Зависимость эквивалентной электропроводности от  концентрации для разбавленных растворов сильных электролитов выражается эмпирическим уравнением Кольрауша:
$$\lambda = \lambda_{0}-A\cdot\sqrt{C},$$
где $A$ -- константа, зависящая от природы растворителя и температуры.

\textbf{Порядок выполнения}

Пипеткой в ячейку наливают 100 мл исследуемого раствора сильного  электролита (электролит берут по  указанию  преподавателя)  и измеряют три раза его сопротивление,  перемешивая перед каждым измерением. Далее раствор разбавляют два раза,  для чего из  стакана (ячейки) отбирают пипеткой 50 мл исследуемого раствора,  переносят его в чистый стакан и добавляют туда этой же пипеткой 50 мл дистиллированной воды. Разбавленный раствор заливают в ячейку и измеряют электропроводность.
Далее измеряют  удельную электропроводность  растворов сильного электролита, например хлорида натрия или хлорида калия,  разбавленных соответственно в 4,  8, 16 и 32 раза (т.е. делают ещё 4 последовательных двойных разбавления  как описано  выше).  Результаты записывают в таблицу \ref{tabular:data32_1}. 
После окончания  работы  с электродами их тщательно промывают дистиллированной водой.

\begin{table}[h]
\caption{Экспериментальные данные}
\label{tabular:data32_1}
\begin{center}
\begin{tabular}{|p{0.2\linewidth}|p{0.2\linewidth}|p{0.2\linewidth}|p{0.25\linewidth}|}
\hline
\No\ измерения & Концентрация электролита, $$C,$$ моль-экв/л & $$C^{-1},$$ л/моль-экв & Удельная электропроводность $$\kappa,$$ См/см \\
\hline
& & & \\
\hline
\end{tabular}
\end{center}
\end{table}

\textbf{Обработка экспериментальных данных}

Измерив удельную электропроводность всех  растворов,  рассчитывают для них значения нормальной концентрации $C$, разведение $C^{-1}$  и результат расчетов вносят в таблицу \ref{tabular:data32_1}.

Постройте график зависимости удельной электропроводности от разведения в координатах $\kappa$ --- $C^{-1}$.

Затем выполняют расчет значений $\sqrt{C}$, эквивалентной электропроводности, $\lambda$, и коэффициента электропроводности $f$. Результаты расчетов вносят в таблицу \ref{tabular:data32_2}.

\begin{table}[h]
\caption{Результаты рассчетов}
\label{tabular:data32_2}
\begin{center}
\begin{tabular}{|p{0.15\linewidth}|p{0.1\linewidth}|p{0.2\linewidth}|p{0.2\linewidth}|p{0.15\linewidth}|}
\hline
\No\ измерения & $$\sqrt{C}$$ & $$C^{-1},$$ л/моль-экв &Эквивалентная электропроводность $$\lambda,$$ См см$^{2}$/моль-экв & Коэффициент электропроводности $$f$$\\
\hline
& & & & \\
\hline
\end{tabular}
\end{center}
\end{table}

Нормальную концентрацию растворов рассчитывают по её исходному значению c учетом двукратных последовательных разбавлений в $a$ раз ($a=0,5$).
$$C_{0}=\frac{n_{0}}{V}$$
При разбавлении:
$$n_{1}=C_{0}\cdot V\cdot a$$
Новая концентрация разбавленного в $a$ раз раствора равна:
$$C_{1}=\frac{n_{1}}{V}=C_{0}\cdot a$$
или:
$$C_{i}=\frac{n_{1}}{V}=C_{0}\cdot a^{i}$$

Эквивалентную электропроводность расчитывают по формуле:
$$\lambda=\frac{\kappa\cdot 1000}{C}$$

Зависимость эквивалентной электропроводности от  концентрации для разбавленных растворов сильных электролитов выражается эмпирическим уравнением Кольрауша:
$$\lambda = \lambda_{0}-A\cdot\sqrt{C},$$
где $A$ -- константа, зависящая от природы растворителя и температуры.

Уравнение Кольрауша является уравнением прямой, не проходящей через начало координат. Постройте график в координатах $\lambda$ --- $\sqrt{C}$. Путем экстраполяции прямой $\lambda=\lambda (\sqrt{C})$ при $C \rightarrow 0$ определите эквитвалентную электропроводность предельно разбавленного раствора $\lambda_{0}$. Сравните это значение с рассчитанным по закону Кольрауша:
$$\lambda_{0}=u_{+}^{0}+u_{-}^{0},$$
где $u_{+}^{0}$ и $u_{-}^{0}$ -- подвижности катионов и анионов (находят по таблице \ref{tabular:data31_1}).

Коэффициент электропроводности $f$ рассчитывают по формуле:
$$f=\frac{\lambda}{\lambda_{0}}$$

Затем строят график зависимости коэффициента электропроводности от концентрации в координатах $f$ --- $C$.

Сделайте вывод о зависимости удельной  и эквивалентной электропроводности от концентрации электролита.  Сделайте вывод о зависимости коэффициента электропроводности от концентрации электролита.

\textbf{Контрольные вопросы}
\begin{enumerate}
\item Зависимость электропроводности от температуры.
\item Тормозящие эффекты в сильных электролитах. 
\item Что такое ионная сила электролита?
\item Что такое активность, подвижность ионов. Средний коэффициент активности электролита.
\item Что учитывает коэффициент электропроводности?
\end{enumerate}

