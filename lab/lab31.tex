\section{Лабораторная работа 5 }
\textbf{Тема:}Определение константы диссоциации слабого электролита.

\textbf{Цель работы:} Определение константы диссоциации слабого электролита методом измерения электропроводности раствора. Изучение влияния концентрации электролита на величину константы диссоциации.

\textbf{Оборудование и реактивы:} кондуктометр, мерный цилиндр на 100 мл, мерные колбы на 100 мл, 1 М раствор $CH_{3}COOH$, стакан, вода.

\textbf{Теория}
\textit{\textbf{1. Проводники и диэлектрики}}
Способность вещества  проводить  электрический  ток  называется электропроводностью. Электропроводность является величиной, обратной электрическому сопротивлению (способности вещества препятствовать прохождению электрического тока).  Характер и количество частиц носителей зарядов (ионов,  электронов) зависит от природы вещества,  его агрегатного состояния, температуры и других факторов, а электрическое поле определяется разностью электрических потенциалов между полюсами и способностью вещества к ионизации.  Вещества различают по проводимости электрического тока.  Строго говоря, все вещества способны проводить электрический ток,  но в практических целях их делят на проводники и непроводники (диэлектрики или  изоляторы),  так как первые обладают в миллионы раз большей проводимостью,  чем вторые. В зависимости от природы частиц носителей зарядов  различают два рода проводников.  К проводникам первого рода относят графит, металлы в твердом и расплавленном состоянии, когда носителями зарядов являются электроны.
К проводникам второго рода относят растворы электролитов,  газы при достаточной степени ионизации и растворы некоторых солей, когда носителями зарядов являются ионы, перемещение которых определяет электро-проводность. Непроводниками считают слюду, фарфор, химически чистую воду,  кристаллы солей, многие низко- и высокомолекулярные  органические соединения и другие.  Некоторые непроводники, приобретающие при изменении внешних условий (температурное, световое, радиационное или другое воздействие) проводимость, называются полупроводниками.  Проводимость в этих веществах возникает, потому что  электроны  переходят в свободное состояние за счет увеличения интенсивности теплового движения частиц тока или за счет положения энергии.  Такими свойствами обладают элементы:  $B$, $C$, $Si$, $Ge$, $Te$, $P$, $S$, $Se$; сплавы: $Mg_{3}Sb_{2}$, $ZnSb$, $InSb$ и др.; оксиды: $Al_{2}O_{3}$, $Cu_{2}O$ и др.; сульфиды : $Cu_{2}S$, $Ag_{2}S$, $ZnS$ и др.; селениды и множество других более сложных соединений.

\textit{\textbf{2. Электропроводимость растворов электролитов}}

Электролиты --- химические соединения,  которые в  растворе  полностью или частично диссоциируют на ионы. Различают сильные и слабые электролиты. Силу электролита характеризуют величиной, названной Арреииусом степенью электрической диссоциации $\alpha$. Под степенью диссоциации понимают соотношение числа молекул, распавшихся на ионы, к общему числу молекул в растворе. Сильные электролиты практически полностью диссоциированы на ионы ($\alpha\approx 1$). К ним относятся многие неорганические соли и кислоты. Для слабых электролитов степень диссоциации очень мала ($\alpha<1$).
Классическая теория  электролитической диссоциации,  созданная Аррениусом (1887),  применена только к разбавленным растворам слабых электролитов. Эта теория исходит из представления, что в растворе электролита существует динамическое равновесие между недиссоциированными молекулами и ионами. Например для уксусной кислоты:
$$CH_{3}COOH\leftrightarrows H^{+}+CH_{3}COO^{-}$$
Количественно это  равновесие можно охарактеризовать константой равновесия, она же константа диссоциации:
$$K=\frac{C_{CH_{3}COO^{-}}\cdot C_{H^{+}}}{C_{CH_{3}COOH}}$$
Константа диссоциации зависит от природы растворенного вещества и растворителя, а так же от температуры.
Степень диссоциации ($\alpha$) и константа диссоциации связаны уравнением Оствальда.
$$K=\frac{\alpha^{2}\cdot C}{1-\alpha}$$
Константа диссоциации  может  быть  определена экспериментально различными методами.  Наиболее распространенным  методом  является метод измерения электрической проводимости растворов.

Под электрической  проводимостью $L$ (электропроводимостью) понимается величина, обратная сопротивлению ($R$): $L = R^{-1}$.

В системе СИ единицей электрической проводимости является сименс (См),  равный  электрической  проводимости проводника, имеющего сопротивление 1 Ом, т.е. [См] = [1/Ом].

Электропроводимость растворов зависит от концентрации ионов, от природы  электролита  и растворителя,  от скорости движения ионов, вязкости и диэлектрической постоянной растворителя.

Различают удельную  и  эквивалентную  проводимость  растворов электролитов.

Удельной электрической проводимостью $\kappa$ ($[\kappa]=$См/см) называют электрическую проводимость раствора,  заключенного между  электродами, площадью по 1 см$^{2}$ и расстоянием 1 см между ними.

Эквивалентная электрическая проводимость $\lambda$ ($[\lambda]=$См см$^{2}$ экв$^{-1}$) равна электрической проводимости  раствора, содержащего  один  эквивалент рас-творенного вещества между электронами, расстояние между которыми равно 1 см.

Площадь электрода определяется объемом раствора. Чем больше разбавление, тем больше  объем  раствора,  и  тем  больше  площадь электродов. Эквивалентная  проводимость и удельная электропроводимость связаны  между  собой  соотношением:
$$\lambda=\frac{\kappa\cdot 1000}{C},$$
где $C$ -- концентрация (экв/л);

Концентрированные растворы  электролитов имеют малое значение электронной проводимости. С разбавлением раствора эта величина увеличивается и при бесконечном разбавлении достигает максимального и постоянного для каждого электролита значения.  Эта  величина обозначается  и называется эквивалентной электрической проводимостью при бесконечном разбавлении. В соответствии с законом Кольреуша может быть вычислена :
$$\lambda_{0}=u_{+}^{0}+u_{-}^{0},$$
где $u_{+}^{0}$ и $u_{-}^{0}$ -- подвижности катионов и анионов (находят по таблице \ref{tabular:data31_1}).

Эквивалентная электрическая проводимость может быть использована для определения степени и константы  электрической диссоциации. По уравне-нию Аррениуса:
$$\alpha=\frac{\lambda}{\lambda_{0}}$$
Тогда:
$$K=\frac{\lambda^{2}C}{\lambda_{0}(\lambda_{0}-\lambda)}$$

\begin{longtable}[h]{|c|c|c|c|}
\caption{Таблица Предельные подвижности $u^{0}_{i}$ некоторых ионов в водном растворе при 25$^{o}$C [Ом$^{-1}$см$^{2}$/г-экв]\label{tabular:data31_1}}\\
\hline
Катионы	 & $u^{0}_{i}$ & Анионы &$u^{0}_{i}$ \\
\hline\endfirsthead
\endhead
$H^{+}$ & 349,8 & $OH^{-}$ & 198,3 \\
\hline 
$Li^{+}$ & 36,68 & $F^{-}$ & 55,4 \\
\hline 
$Na^{+}$ & 50,10 & $Cl^{-}$ & 76,35 \\
\hline 
$K^{+}$ & 73,50 & $Br^{-}$ & 78,14 \\
\hline 
$Rb^{+}$ & 77,81 & $I^{-}$ & 78,84 \\
\hline 
$Ag^{+}$ & 61,90 & $ClO_{3}^{-}$ & 64,6 \\
\hline 
$NH_{4}^{+}$ & 73,55 & $ClO_{4}^{-}$ & 67,36 \\
\hline 
$N(CH3)_{4}^{+}$ & 44,92 & $BrO_{3}^{-}$ & 55,74 \\
\hline 
$1/2 Mg^{2+}$ & 53,05 & $CN^{-}$ & 78 \\
\hline 
$1/2 Ca^{2+}$ & 59,50 & $NO_{3}^{-}$ & 71,46 \\
\hline 
$1/2 Ba^{2+}$ & 63,63 & $CH_{3}COO^{-}$ & 40,90 \\
\hline 
$1/2 Mg^{2+}$ & 56,6 & $C_{6}H_{5}COO^{-}$ & 35,8 \\
\hline 
$1/2 Cd^{2+}$ & 54 & $H_{2}PO_{4}^{-}$ & 36 \\
\hline 
$1/3 Al^{3+}$ & 63 & $1/2 SO_{4}^{2-}$ & 80,02 \\
\hline 
$1/3 La^{3+}$ & 69,7 & $1/2 S_{2}O_{6}^{2-}$ & 93 \\
\hline 
\end{longtable}

\textbf{Порядок выполнения}
Пипеткой в ячейку наливают 100 мл исследуемого раствора слабого электролита (электролит берут по  указанию  преподавателя)  и измеряют три раза его сопротивление,  перемешивая перед каждым измерением. Далее раствор разбавляют два раза,  для чего из  стакана (ячейки) отбирают пипеткой 50 мл исследуемого раствора,  переносят его в чистый стакан и добавляют туда этой же пипеткой 50 мл дистиллированной воды. Разбавленный раствор заливают в ячейку и измеряют электропроводность.
Далее измеряют  удельную электропроводность  растворов,  разбавленных соответственно в 4,  8, 16 и 32 раза (т.е. делают ещё 4 последовательных двойных разбавления  как описано  выше).  Результаты записывают в таблицу \ref{tabular:data31_2}. 
После окончания  работы  с электродами их тщательно промывают дистиллированной водой.

\begin{table}[h]
\caption{Экспериментальные данные}
\label{tabular:data31_2}
\begin{center}
\begin{tabular}{|p{0.1\linewidth}|p{0.2\linewidth}|p{0.2\linewidth}|p{0.2\linewidth}|p{0.15\linewidth}|}
\hline
\No\ измерения & Концентрация электролита, $C$, моль-экв/л & Удельная электропроводность $\kappa$, См/см & Эквивалентная электропроводность $\lambda$, моль-экв/л & Константа диссоциации $K$\\
\hline
& & & & \\
\hline
\end{tabular}
\end{center}
\end{table}

\textbf{Обработка экспериментальных данных}
Измерив удельную электропроводность всех  растворов,  рассчитывают для них значения нормальной концентрации $C$, эквивалентной электропроводности, $\lambda$, и константу диссоциации $K$. Результаты расчетов вносят в таблицу \ref{tabular:data31_2}.

Нормальную концентрацию рассчитывают по  исходному  значению  её учетом двукратных последовательных разбавлений  в $a$ раз ($a=0,5$).
$$C_{0}=\frac{n_{0}}{V}$$
При разбавлении:
$$n_{1}=C_{0}\cdot V\cdot a$$
Новая концентрация разбавленного в $a$ раз раствора равна:
$$C_{1}=\frac{n_{1}}{V}=C_{0}\cdot a$$
или:
$$C_{i}=C_{0}\cdot a^{i}$$

Эквивалентную электропроводность расчитывают по формуле:
$$\lambda=\frac{\kappa\cdot 1000}{C}$$

Константу диссоциации для одноосновных кислот рассчитывают по формуле:
$$K=\frac{\lambda^{2}C}{\lambda_{0}(\lambda_{0}-\lambda)},$$
где $\lambda_{0}$ -- эквитвалентная электропроводность предельно разбавленного раствора, определяемая по закону Кольрауша:
$$\lambda_{0}=u_{+}^{0}+u_{-}^{0},$$
где $u_{+}^{0}$ и $u_{-}^{0}$ -- подвижности катионов и анионов (находят по таблице \ref{tabular:data31_1}).

Вычислив значение  $K$  для  различных  концентраций слабого электролита, сделать вывод о влиянии концентрации электролита на величину константы диссоциации.

\textbf{Контрольные вопросы}
\begin{enumerate}
\item Что такое электролитическая диссоциация? Виды диссоциации. 
\item Степень диссоциации, константа диссоциации.
\item Подвижность ионов. Закон Кольрауша. Числа переноса ионов.
\item Что такое изотонический коэффициент? Его влияние на законы идеальных растворов.
\item Что такое слабые и сильные электролиты?
\item Закон Оствальда. Вывод уравнения для одноосновных кислот. Записать уравнения закона Оствальда для данного электролита через степень диссоциации и через эквивалентную электропроводность.
\item Закон Оствальда. Вывод уравнения для двухосновных кислот. Записать уравнения закона Оствальда для данного электролита через степень диссоциации и через эквивалентную электропроводность.
\item Для каких электролитов применяют законы разведения?
\item Что такое удельная эквивалентная электропроводимость?
\item От чего зависит электропроводность растворов? 
\item Как, исходя из теории электролитической диссоциации и теории сильных  электролитов,  объяснить  зависимость $K$ и $\lambda$ от разбавления?
\end{enumerate}

