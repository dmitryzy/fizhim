\section{Лабораторная работа 2 }
\textbf{Тема:}Определение коэффициента распределения

\textbf{Цель работы:}Определить коэффициент распределения йода между водой и органическим растворителем -- толуолом.

\textbf{Оборудование и реактивы:} аппарат для встряхивания, бюретка, 4 колбы, с притёртыми пробками на 200 мл, пипетка на 5 мл, мерный цилиндр на 50 мл, раствор йода в толуоле (концентрации: 5, 10, 20 г/л), раствор тиосульфата натрия (концентрация: 0,025 Н).

\textbf{Теория}

Изучение распределения  вещества между двумя несмешивающимися растворителями представляет большой интерес,  так как может  дать ценные сведения, необходимые для проведения экстрагирования, а так же указать на наличие диссоциации,  ассоциации или других химических  реакций  растворенного  вещества в растворе.  Если в систему, состоящую из двух несмешивающихся жидкостей,  ввести небольшое количество вещества,  растворимого в этих жидкостях,  то после установления равновесия, оно распределится между обоими жидкими слоями в определенном, постоянном при данной температуре соотношении.

Экстракцией называют извлечение из многокомпонентного 
раствора одного или нескольких компонентов с помощью растворителя, обладающего избирательной способностью растворять только подлежащий
экстратированию компонент . При помощи экстракции происходит извлечение необходимых веществ: сахара из свёклы, растительного масла из семечек, в формации для алкалоидов и других физиологически активных веществ.	
На основе закона распределения можно рассчитать эффективность экстракции в зависимости от свойств растворителя и экстратирующего вещества.
Экстракция может быть однократной, когда экстратент добавляется в один приём, и дробной – добавления экстратента производится порциями в несколько приёмов.

\textbf{Порядок выполнения}

1. В 4 Колбы на 200 мл с притёртыми пробками отмерить  пипеткой по 5 мл раствора йода различной концентрации и по 50 мл дистиллированной воды.  Полученные смеси энергично встряхивать в течении 30 минут.

2. Перелить содержимое колб в делительные воронки и поставить на 20 минут до расслоения жидкостей.

3. Отделить водный слой, отобрать пипеткой по $V_{\textrm{пр}}=20$ мл из каждой воронки и  оттитровать  йод  0,025 Н раствором тиосульфата натрия ($C_{\textrm{т}}=0,025$моль/л). Раствор тиосульфата прибавляют до появления бледно-желтой  окраски раствора. Затем добавляют  несколько  капель  раствора крахмала и снова титруют раствором тиосульфата натрия до исчезновения  синего окрашивания раствора.  Светло-синяя окраска раствора, появляющаяся через некоторое время после титрования,  не  учитывается.  Пипетку перед отбором пробы ополаскивают исследуемым раствором.

\textbf{Обработка экспериментальных данных}

1. Рассчитать равновесную концентрацию йода (г/л) в водном растворе по формуле:
$$C_{\textrm{в}}=\frac{C_{\textrm{т}}M_{I}V_{\textrm{т}}}{V_\textrm{пр}}$$
Молярная масса йода равна: $M_{I}=127$г/моль.

2. Рассчитать  равновесную  концентрацию йода в толуоле.  Она определяется из уравнения материального баланса для йода при экстракции.
$$m_{0}=m_{\textrm{в}}+m_{\textrm{орг}}$$
$$C_{0}V_{0}=C_{\textrm{в}}V_{\textrm{в}}+C_{\textrm{орг}}V_{\textrm{орг}}$$
$$C_{\textrm{орг}}=\frac{1}{V_{\textrm{орг}}}(C_{0}V_{0}-C_{\textrm{в}}V_{\textrm{в}})$$
При объеме водной фазы $V_{\textrm{в}}=50$мл, объеме органической фазы, раствора йода в толуоле, ($V_{0}=V_{\textrm{орг}}=5$мл) равновесная концентрация йода в толуоле равна:
$$C_{\textrm{орг}}=C_{0}-10C_{\textrm{в}}$$

3. Рассчитать значение константы распределения по формуле:
$$K=\frac{C_{\textrm{орг}}}{C_{\textrm{в}}}=\frac{C_{0}}{C_{\textrm{в}}}-10$$

Все экспериментальные и расчетные данные  сводим  в  таблицу \ref{tabular:data2}.

\begin{table}[h]
\caption{Экспериментальные и расчетные данные}
\label{tabular:data2}
\begin{center}
\begin{tabular}{|c|p{3cm}|p{3cm}|p{3cm}|p{3cm}|p{3cm}|}
\hline
\No & Исходные концентрации йода в толуоле, $C_{0}$, г/л & Объём пробы для титрования, $V_{\textrm{пр}}$, мл & Объем 0,025 Н $Na_{2}S_{2}O_{3}$ пошедшего на титрование, $V_{\textrm{т}}$, мл & Равновесная концентрация йода в воде, $C_{\textrm{в}}$, г/л & Коэффициент распределения, $K$ \\
\hline
1 & 5 & 20 & & & \\
\hline
2 & 10 & 20 & & & \\
\hline
3 & 20 & 20 & & & \\
\hline
\end{tabular}
\end{center}
\end{table}

\textbf{Контрольные вопросы}
\begin{enumerate}
\item Что называется фазой, компонентом и степенью свободы? 
\item В чем заключается физико-химический метод анализа? 
\item На чем основан термический анализ ?
\item Как изображается состав трехкомпонентной системы по методу Гиббса?
\item В чем заключается процесс экстрагирования,  какова его теоретическая основа?
\item Что такое мольная доля?
\item Что такое химический потенциал?
\item Закон распределения.
\item Правило фаз Гиббса.
\end{enumerate}