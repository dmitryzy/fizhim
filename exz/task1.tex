%Задачи по физической химии
%к экзамену
%
%1-е задачи
\Qvery 
Рассчитайте энтальпию образования $N_{2}O_{5}$ на основании следующих данных:
$$2NO(\textsc{г})+O_{2}(\textsc{г})=2NO(\textsc{г}) \Delta H_{1}^{0}=-114,2 \textsc{кДж/моль}$$
$$4NO_{2}(\textsc{г})+O_{2}(\textsc{г})=2N_{2}O_{5}(\textsc{г}) \Delta H_{2}^{0}=-110,2  \textsc{кДж/моль}$$
$$N_{2}(\textsc{г})+O_{2}(\textsc{г})=2NO(\textsc{г}) \Delta H_{3}^{0}=182,6 \textsc{кДж/моль}$$
\endQvery
\Qvery 
Найдите $\Delta_{r}H_{298}^{0}$ для реакции:
$$CH_{4}(\textsc{г})+Cl_{2}(\textsc{г})=CH_{3}Cl(\textsc{г})+HCl(\textsc{г})$$
если известны: 

$$\Delta_{c}H_{298}^{0}(CH_{4})=-890,6 \textsc{кДж/моль},$$
$$\Delta_{c}H_{298}^{0}(CH_{3}Cl)=-689,8 \textsc{кДж/моль},$$
$$\Delta_{c}H_{298}^{0}(H_{2})=-285,8 \textsc{кДж/моль},$$
$$\Delta_{f}H_{298}^{0}(HCl)=-92,3 \textsc{кДж/моль}.$$
\endQvery
\Qvery 
Рассчитайте стандартное изменение энтропии в реакции:
$$H_{2}(\textsc{г})+0,5O_{2}(\textsc{г})=H_{2}O(\textsc{г})$$
Справочные данные: 
$$S^{0}_{298}(H_{2})=130,684\textsc{Дж/(мольК)},$$
$$S^{0}_{298}(O_{2})=205,138\textsc{Дж/(мольК)},$$
$$S^{0}_{298}(H_{2}O)=188,83\textsc{Дж/(мольК)}.$$
\endQvery
\Qvery 
Пользуясь принципом Ле-Шателье проанализировать влияние изменений давления, температуры, концентраций реагентов на смещение равновесия системы.
$$CH_{4}(\textsc{г})+CO_{2}(\textsc{г})=2CO(\textsc{г})+2H_{2}(\textsc{г})$$ 
\endQvery
\Qvery 
Пользуясь принципом Ле-Шателье проанализировать влияние изменений давления, температуры, концентраций реагентов на смещение равновесия системы.
$$4NH_{3}(\textsc{г})+5O_{2}(\textsc{г})=4NO(\textsc{г})+6H_{2}O(\textsc{г})$$
\endQvery
\Qvery 
Записать уравнения закона действующих масс для скоростей прямой и обратной реакций, выражение константы равновесия через парциальные давления компонентов для системы:
$$CH_{4}(\textsc{г})+CO_{2}(\textsc{г})=2CO(\textsc{г})+2H_{2}(\textsc{г})$$ 
Частные порядки реакции по реагентам принять равными стехиометрическим коэффициентам.
Рассчитайте отношение $\frac{K_{p}}{K_{x}}$ при нормальном давлении.
\endQvery
\Qvery 
Записать уравнения закона действующих масс для скоростей прямой и обратной реакций, выражение константы равновесия через парциальные давления для системы:
$$4NH_{3}(\textsc{г})+5O_{2}(\textsc{г})=4NO(\textsc{г})+6H_{2}O(\textsc{г})$$
Частные порядки реакции по реагентам принять равными стехиометрическим коэффициентам.
Рассчитайте отношение $\frac{K_{p}}{K_{x}}$ при нормальном давлении.
\endQvery
\Qvery 
Вычислите тепловой эффект реакции 
$$FeO(\textsc{к}) + H_{2} (\textsc{г})= Fe(\textsc{к}) + H_{2}O(\textsc{г}) +Q$$
исходя из следующих термохимических уравнений:
$$FeO(\textsc{к})+CO(\textsc{г})= Fe(\textsc{к})+CO_{2} (\textsc{г})+13,2 \textsc{кДж/моль}.$$
$$CO(\textsc{г})+0,5O_{2} (\textsc{г})= CO_{2} (\textsc{г})+283,0 \textsc{кДж/моль}.$$
$$H_{2} (\textsc{г})+0,5O_{2} (\textsc{г})= H_{2}O(\textsc{г})	+241,8 \textsc{кДж/моль}.$$
\endQvery
\Qvery 
Записать выражения констант диссоциации серной, соляной и уксусной кислот и уравнения электролитической диссоциации указанных кислот. Как изменятся численные значения указанных констант диссоциации при постоянной температуре 25$^{o}C$ и повышении их концентраций соответствующих кислот от 1 моль/л до 2 моль/л?
\endQvery
\Qvery 
 Пользуясь правилом Ле-Шателье определите, какая из приведенных систем не реагирует на изменение давления.
\begin{enumerate}
\item $CH_{4}(\textsc{г})+CO_{2}(\textsc{г}) \rightleftarrows 2CO(\textsc{г})+2H_{2}(\textsc{г})$; 
\item $FeO(\textsc{т})+CO(\textsc{г}) \rightleftarrows Fe(\textsc{т})+CO_{2}(\textsc{г})$;
\item $4NH_{3}(\textsc{г})+5O_{2}(\textsc{г}) \rightleftarrows 4NO(\textsc{г}) +6H_{2}O(\textsc{г})$; 
\item $O_{2}(\textsc{г})+4HCl (\textsc{ж})\rightleftarrows 2H_{2}O(\textsc{г})+2Cl_{2}(\textsc{г})$.
\end{enumerate} 
\endQvery
\Qvery 
Зависимость энтальпии от температуры. Написать выражение и нарисовать график зависимости $\Delta H$ от $T$ при $C_{p}\approx 0$, $C_{p}=0$, $C_{p}<0$ (считать теплоемкость постоянной).
\endQvery
\Qvery 
Установить характер реакции в системе 
$$2HI \leftrightarrows H_{2}+I_{2}$$
Константы равновесия при температурах 573 и 636 K равны $1,3 \cdot 10^{-2}$ и $1,68 \cdot 10^{-2}$.
\endQvery
\Qvery 
В состоянии равновесия в системе 
$$N_{2} + 3H_{2}  \leftrightarrows 2NH_{3}$$
при стандартных условиях энергия Гиббса равна 25 кДж/моль. Вычислить константу равновесия реакции при 900 K.
\endQvery
\Qvery 
Найдите минимальную температуру, при которой возможно протекание реакции:
$$CO(\textsc{г})+H_{2}O(\textsc{ж})=CO_{2}Cl(\textsc{г})+H_{2}(\textsc{г})$$
если известны: 

$$\Delta H_{298}^{0}(CO_{2})=-393,5 \textsc{кДж/моль},$$
$$\Delta H_{298}^{0}(CO)=-110,52 \textsc{кДж/моль},$$
$$\Delta H_{298}^{0}(H_{2}O)=-285,84 \textsc{кДж/моль},$$

$$\Delta S_{298}^{0}(CO_{2})=213,2 \textsc{кДж/моль},$$
$$\Delta S_{298}^{0}(CO)=197,9 \textsc{кДж/моль},$$
$$\Delta S_{298}^{0}(H_{2}O)= 188,72 \textsc{кДж/моль},$$
$$\Delta S_{298}^{0}(H_{2})=130,59 \textsc{кДж/моль}.$$
\endQvery
\Qvery 
Прямая или обратная реакция будет протекать в системе при 800 К?
$$CO(\textsc{г})+H_{2}O(\textsc{ж})=CO_{2}Cl(\textsc{г})+H_{2}(\textsc{г})$$
Справочные данные: 

$$\Delta H_{298}^{0}(CO_{2})=-393,5 \textsc{кДж/моль},$$
$$\Delta H_{298}^{0}(CO)=-110,52 \textsc{кДж/моль},$$
$$\Delta H_{298}^{0}(H_{2}O)=-285,84 \textsc{кДж/моль},$$

$$\Delta S_{298}^{0}(CO_{2})=213,2 \textsc{кДж/моль},$$
$$\Delta S_{298}^{0}(CO)=197,9 \textsc{кДж/моль},$$
$$\Delta S_{298}^{0}(H_{2}O)= 188,72 \textsc{кДж/моль},$$
$$\Delta S_{298}^{0}(H_{2})=130,59 \textsc{кДж/моль}.$$
\endQvery
\Qvery 
Рассчитайте константу равновесия реакции при при 1000 К?
$$CO(\textsc{г})+H_{2}O(\textsc{ж})=CO_{2}Cl(\textsc{г})+H_{2}(\textsc{г})$$
Справочные данные: 

$$\Delta H_{298}^{0}(CO_{2})=-393,5 \textsc{кДж/моль},$$
$$\Delta H_{298}^{0}(CO)=-110,52 \textsc{кДж/моль},$$
$$\Delta H_{298}^{0}(H_{2}O)=-285,84 \textsc{кДж/моль},$$

$$\Delta S_{298}^{0}(CO_{2})=213,2 \textsc{кДж/моль},$$
$$\Delta S_{298}^{0}(CO)=197,9 \textsc{кДж/моль},$$
$$\Delta S_{298}^{0}(H_{2}O)= 188,72 \textsc{кДж/моль},$$
$$\Delta S_{298}^{0}(H_{2})=130,59 \textsc{кДж/моль}.$$
\endQvery
\Qvery 
В состоянии равновесия в системе 
$$N_{2} + 3H_{2}  \leftrightarrows 2NH_{3}$$
при стандартных условиях константа равновесия равна $1,7 \cdot 10^{-4}$. Вычислить энергию Гиббса при стандартных условиях.
\endQvery
%
%2-е задачи
\Qvery 
Удельная электропроводность 4\% водного раствора $H_{2}SO_{4}$ при 18$^{0}$C равна 0,168~См/см. плотность раствора  1,026 г/см$^{3}$. Рассчитать эквивалентную электропроводность раствора. 
\endQvery
\Qvery 
Константа диссоциации гидроксида аммония равна $1,79\cdot 10^{-5}$~моль/л. Рассчитать концентрацию $NH_{4}OH$, при которой степень диссоциации равна 0,01 и эквивалентную электропроводность раствора при этой концентрации. Предельные подвижности ионов в водном растворе равны: 

$\lambda_{NH_{4}^{+}}^{0}=73,55$ См~см$^{2}$/моль; 

$\lambda_{OH^{-}}^{0}=198,3$ См~см$^{2}$/моль.
\endQvery
\Qvery 
Рассчитать константу равновесия реакции диспропорционирования
$$2Cu^{+}=Cu^{2+}+Cu$$
при 25~$^{o}$C по следующим данным:
$$\varphi_{Cu^{2+}/Cu^{+}}=0,153 B $$
$$\varphi_{Cu^{+}/Cu}=0,521 B$$
\endQvery
\Qvery 
Рассчитайте константу равновесия реакции
$$ZnSO_{4}+Cd=CdSO_{4}+Zn$$
по следующим данным:
$$\varphi_{Zn^{2+}/Zn}=-0,763 B $$
$$\varphi_{Cd^{2+}/Cd}=-0,403 B$$
\endQvery
\Qvery 
В гальваническом элементе обратимо протекает реакция
$$CuSO_{4}+Zn=ZnSO_{4}+Cu$$
Рассчитайте $\Delta H$ и $\Delta S$ реакции, если ЭДС элемента равна 1,096~В при 273~К и 1,0961~В при 276~К.
\endQvery
\Qvery 
С помощью правила Вант-Гоффа вычислите, при какой температуре реакция закончится через 15 минут, если  при 20$^{o}C$ на это требуется 2 часа. температурный коэффициент скорости равен 3.
\endQvery
\Qvery 
Энергия активации некоторой реакции в 1,5 раз больше, чем энергия активации другой реакции. При нагревании от $T_{1}$ до $T_{2}$ константа скорости второй реакции увеличилась в 2 раза. Во сколько раз увеличится константа скорости первой реакции при нагревании от  $T_{1}$ до $T_{2}$.
\endQvery
\Qvery 
Измерены константы скорости некоторой реакции 2-го порядка при разных температурах. При 350~$^{o}C$ $k=1,57$~л/(моль с).  При 400~$^{o}C$ $k=7,73$~л/(моль с). Определите параметры уравнения Аррениуса.
\endQvery
\Qvery 
Энергия активации некоторой реакции в 1,5 раз больше, чем энергия активации другой реакции. При нагревании от $T_{1}$ до $T_{2}$ константа скорости второй реакции увеличилась в 4 раза. Во сколько раз увеличится константа скорости первой реакции при нагревании от  $T_{1}$ до $T_{2}$.
\endQvery