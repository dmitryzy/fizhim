%Вопросы для сдачи экзамена по курсу «Коллоидная химия» (для группы 31 ИЗ)
%Классификация
\section{Вопросы по коллоидной химии}
\begin{enumerate}
\item 
Классификация  дисперсных систем. Лиофильные и лиофобные дисперсные системы. 
 
\item 
Методы получения и очистки дисперсных систем.
 
\item 
Коллоидная химия и экологические проблемы гидросферы.
 
\item 
Коллоидная химия и экологические проблемы биосферы.
 
\item 
Коллоидная химия и экологические проблемы атмосферы.
 
\item 
Коллоидная химия и экологические проблемы литосферы.
 
%
%Поверхностное натяжение
\item 
Термодинамические характеристики поверхности. Температурная зависимость поверхностного натяжения.
 
\item 
Поверхностное натяжение. Правило Антонова.
 
\item 
Смачивание. Закон Юнга. Краевой угол; термодинамические условия смачивания и растекания. Влияние ПАВ на краевые углы.
 
\item 
Адгезия и когезия.
 
\item 
Капиллярное давление. Закон Лапласа и его следствия.
 
\item 
Влияние кривизны поверхности (размера частиц) на давление насыщенного пара и растворимость вещества. Изотермическая перегонка и капиллярная конденсация.
 
\item 
Методы измерения поверхностного натяжения и свободной поверхностной энергии твердых тел.
 
%
%Адсорбция
\item 
Адсорбция. Единицы измерения. Физическая и химическая адсорбция. Конкурентная адсорбция компонентов раствора.
 
\item 
Связь между адсорбцией и поверхностным натяжением.
 
\item 
Основы термодинамики адсорбции на поверхности раздела жидкость/газ. Уравнение Гиббса.
 
\item 
Адсорбция растворимых ПАВ на поверхности раздела раствор ПАВ/воздух. Связь уравнений Гиббса, Ленгмюра и Шишковского.
 
%
%ДЭС
\item 
Причины образования двойного электрического слоя (ДЭС). Современные представления о строении ДЭС.
 
\item 
Плотная и диффузная части ДЭС. Изменение потенциала в двойном электрическом слое для сильно и слабо заряженных поверхностей.
 
\item 
Влияние электролитов на строение ДЭС. Ионный обмен в дисперсных системах.
 
%
%ПАВ
\item 
Строение адсорбционных слоев поверхностно-активных веществ. 
 
\item 
Поверхностная активность. Вещества поверхностно-активные (ПАВ) и инактивные (ПИАВ) по отношению к воде и другим растворителям.
 
\item 
Поверхностная активность. Правило Дюкло-Траубе.
 
%
%Электрокинетические явления
\item 
Классификации электрокинетических явлений. Электроосмос. Потенциал оседания. Потенциал течения.
 
\item 
Электрокинетический ($\zeta$) потенциал и вариации его изменения при введении в раствор электролитов. 
 
\item 
Электрофорез. Эффекты, осложняющие электрофорез. 
 
%
%Устойчивость дисперсных систем
\item 
Понятия седиментационной и агрегативной устойчивости лиофобных дисперсий, факторы их определяющие.
 
\item 
Коагуляция. Причины, разрушения дисперсных систем. 
 
\item 
Коагуляция лиофобных золей электролитами. Порог коагуляции. Правило Шульце-Гарди.
 
\item 
Изменение электрокинетического потенциала частиц при коагуляции индифферентными и неиндифферентными электролитами. 
 
\item 
Основные идеи теории коагуляции гидрофобных золей электролитами (теория ДЛФО). 
 
\item 
Теория кинетики быстрой коагуляции Смолуховского.
 
\item 
Самопроизвольное диспергирование. Критерий Ребиндера и Щукина.
 
%
%Дисперсные системы
\item 
Порядок работы с колориметром при определении оптической плотности коллоидных растворов.
 
\item 
Мицеллообразование в водных средах. Термодинамика мицеллообразования.
 
\item 
Пены. Получение и строение. Устойчивость пен. Основные применения.
 
\item 
Порошки. Получение и строение. Основные применения.
 
\item 
Эмульсии. Классификация эмульсий. Методы определения типа эмульсий. Устойчивость и обращение фаз в эмульсиях.
 
\item 
Стабилизация эмульсий и обращение фаз. Принцип подбора эмульгаторов. 
 
\item 
Седиментационная и агрегативная устойчивость дисперсных систем. Факторы агрегативной устойчивости дисперсных систем.
 
\item 
Структурно-механический барьер по Ребиндеру как фактор устойчивости дисперсных систем.
 

\end{enumerate}