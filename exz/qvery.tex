%%Вопросы для экзамена 
\section{Вопросы по физической химии}
\subsection{Термодинамика. Химическое равновесие}
\begin{enumerate}
\item 
Основные понятия химической термодинамики: система, фаза, компонент. Виды термодинамических систем и процессов. 

\item 
Тепловой эффект химической реакции. Экзо- и эндотермические реакции. Стандартная теплота образования? Стандартная теплота сгорания? Закон Гесса. Следствия закона Гесса. Написать выражения.
 
\item 
Нулевой закон термодинамики. Первый закон термодинамики. Его формулировка и следствия. 

\item 
Первое начало термодинамики для изобарного, изохорного и изотермического процессов. Привести уравнения. Внутренняя энергия как термодинамическая функция.
 
\item 
Энтальпия как термодинамическая функция. Единицы измерения. Энтальпии образования химических соединений.
 
\item 
Теплоемкость. Определение теплоемкости в классической термодинамике. Единицы измеряения. Зависимость энтальпий химических реакций от температуры. Уравнение Кирхгофа (в интегральной и дифференциальной формах) для изобарного и изохорного процессов.
 
\item 
Теплоты реакций $Q_{V}$ и $Q_{p}$. Экзотермические и эндотермические реакции. Стандартные энтальпии химических реакций. 
 
\item 
Формулировка закона Гесса. Первое, второе следствия закона Гесса.
 
\item 
Энергия Гиббса как критерий самопроизвольного протекания процесса. Стандартная энергия Гиббса химической реакции. Единицы измерения..
 
\item 
Энергия Гельмгольца как термодинамическая функция. Критерий протекания самопроизвольных процессов в изохорно-изотермических условиях. 
 
\item 
Второй закон термодинамики. Энтропия, как функция состояния. Изменение энтропии при необратимых процессах. Статистическая природа второго начала термодинамики.  Термодинамическая вероятность.

\item 
Объединенное уравнение 1-го и 2-го загонов термодинамики. Уравнение Гиббса-Дюгема. Учет химических реакций.

\item 
Уравнения Гиббса-Гельмгольца.

\item
Метод термодинамических потенциалов. Уравнения Максвелла. 

\item 
Третий закон термодинамики. Расчет энтропии.
 
\item 
Химический потенциал. Его различные определения. Химический потенциал идеального газа. Выражение химического потенциала через энергию Гиббса, энтальпию,  энергию Гельмгольца и энтропию.
 
\item 
Химические равновесия в закрытых системах. Условие химического равновесия. Изотерма химической реакции. Различные формы записи констант равновесия и связь между ними. 
 
\item 
Фазовое равновесие. Степень свободы, компонент, фаза. Правило фаз Гиббса. Условия фазового равновесия. 

\item 
Фазовые равновесия в однокомпонентных системах. Уравнение Клапейрона Клаузиуса. Его применение к процессам плавления, сублимации и испарения в однокомпонентных системах (на примере $H_{2}O$). Правило фаз Гиббса.

\item
Закон распределения. Дать формулировку, записать уравнение. Что такое коэффициент распределения и от чего он зависит?

\item 
Равновесие жидкость - пар в двухкомпонентных системах. Фазовые диаграммы двукомпонентных систем. Азеотропные смеси. Законы Коновалова.

\end{enumerate}

\subsection{Адсорбция}
\begin{enumerate}
\item 
Адсорбция. Единицы измерения. Физическая и химическая адсорбция.

\item
Избирательная адсорбция. Лиотропные ряды. Правило Панета-Фаянса.

\item
Что такое ионообменная адсорбция? Иониты и их виды.

\item
Адсорбция электролитов. Механизмы образования двойного электрического слоя (ДЭС). Строение ДЭС. Влияние многозарядных ионов на строение ДЭС. Что такое сверхэквивалентная адсорбция?

\item
Полимолекулярная адсорбция. Теория БЭТ. Теория Поляни.

\item
Мономолекулярная адсорбция. Изотерма адсорбции.  Изотерма Лэнгмюра (график). Уравнение Генри (график). Дать объяснения.

\item 
Поверхностная активность. Вещества поверхностно-активные (ПАВ) и инактивные (ПИАВ) по отношению к воде и другим растворителям.

\item
Дать анализ уравнения Гиббса. ПАВ и ПИВ. Их влияние на адсорбцию. Поверхностное натяжение. Полярность. Правило Ребиндера. 

\item 
Поверхностная активность. Правило Дюкло-Траубе.

\end{enumerate}

\subsection{Растворы. Дисперсные системы}
\begin{enumerate}
\item 
Идеальные растворы. Закон Рауля и закон Генри. 
\item
Основные особенности и классификация дисперсных систем (по агрегатному состоянию и по размеру частиц). Теория устойчивости дисперсных систем. Потенциальная кривая зависимости сил взаимодействия между частицами. Природа сил отталкивания и притяжения.
  
\item 
Классификация  дисперсных систем. Лиофильные и лиофобные дисперсные системы. 

\item
Получение дисперсных систем методом конденсации. Привести условия получения. Физическая конденсация. Основные методы. Условия образования зародышей новой фазы, что такое степень пересыщения? Дать описание.

\item
Химическая конденсация, реакция окисления, гидролиза, двойного обмена. Привести реакции и схемы мицелл.

\item
Получение коллоидных растворов методом диспергирования, теория Ребиндера. Эффект Ребиндера и его механизм. Физико-химическое диспергирование. Метод адсорбционной пептизации, привести пример и схему мицеллы.

\item
Физико-химическая пептизация. Метод промывания осадка. Привести пример реакции и схему мицеллы. Дать объяснение. Правило осадков Оствальда. Зависимость золя пептизированного осадка от концентрации электролита. Привести рисунок. Дать объяснение.

\item
Строение коллоидных частиц. Что такое мицелла (пример), интермицелярная жидкость. Что такое противоионы? Свободные и связанные противоионы (привести пример). Что такое адсорбционный и диффузионный слои, гранула (привести примеры). 

\item
Агрегативная устойчивость дисперсных систем. Кинетические факторы устойчивости. Что такое седиментационная устойчивость коллоидных систем? Каковы основные условия этой устойчивости? Термодинамические факторы устойчивости.

\item 
Строение коллоидных частиц. Образование двойного электрического слоя (ДЭС) по Гельмгольцу (привести схему). Образование ДЭС по Штерну (привести схему). Влияние одно-, двух-, трех-, четырехзарядных электролитов на строение ДЭС. Дать объяснение.Электрокинетические явления. Электрофорез. Электроосмос. Потенциал седиментации. Потенциал течения.

\item 
Теория ДЛФО. Показать поведение коллоидных частиц в случае отсутствия коагуляции и когда наступает коагуляция. Дать объяснение (рисунок). Что такое потенциальный барьер коагуляции? Показать поведение коллоидных систем в случае образования структурированных систем (рисунок).

\item 
Механизмы коагуляции золей: концентрационный и адсорбционный. Коагуляция. Коагуляция золей при действии электролита. Правило Шульце-Гарди. Что такое порог коагуляции? Коагулирующая способность. Влияние размера иона коагулятора на коагуляцию. 

\item 
Влияние заряда иона коагулятора на коагуляцию индеферентного электролита. Скорость коагуляции. Кинетика быстрой коагуляции Смолуховского (схема).

\item 
Коагуляция золей смесями электролитов. Что такое аддитивность, антогонизм и синергизм электролитов. Их механизм. Привыкание золей. Положительное и отрицательное привыкание золей и их механизм (рисунок). Защита коллоидных растворов от коагуляции. Механизм защитного действия. Солюбилизация.

\item 
Классификации электрокинетических явлений. Электроосмос. Потенциал оседания. Потенциал течения.
 
\item 
Электрофорез. Эффекты, осложняющие электрофорез. 

\item 
Понятия седиментационной и агрегативной устойчивости лиофобных дисперсий, факторы их определяющие.
 
\item 
Коагуляция. Причины, разрушения дисперсных систем. Порог коагуляции. Правило Шульце-Гарди.

\end{enumerate}

\subsection{Электрохимия. Растворы элекролитов}
\begin{enumerate}
\item 
Что такое электролитическая диссоциация? Степень диссоциации, константа диссоциации и их связь. Что такое изотонический коэффициент? Его влияние на законы идеальных растворов.

\item
Сильные  электролиты. Что такое ионная сила электролита, ионнная атмосфера (привести рисунок).  Что такое активность, подвижность ионов. Коэффициент активности. Электрофоретический эффект торможения. Релаксационный эффект. 

\item
Эквивалентная электропроводность. Дать понятие. В каких единицах измеряется? Что такое разведение электролита, предельная эквивалентная электропроводность?

\item 
Удельная электропроводность. В каких единицах измеряется? Зависимость удельной электропроводности для сильных и слабых электролитов (привести схемы).
 
\item 
Сильные и слабые электролиты. Степень диссоциации. Константа диссоциации.
 
\item 
Электролиз. Законы Фарадея.

\item
Уравнение Нернста для медно-цинкового гальванического элемента и для водородного электрода. Привести реакции.

\item 
Электродный потенциал. Механизм возникновения электродного потенциала. Возникновение двойного электрического слоя (ДЭС) на границе металл-раствор.
\end{enumerate}

\subsection{Химическая кинетика}
\begin{enumerate}
\item 
Скорость химической реакции. Константа скорости. Физический смысл константы скорости реакции. Закон действующих масс. Влияние концентраций реагентов на скорость химической реакции.
 
\item 
Влияние температуры на скорость химической реакции. Уравнение Аррениуса. Энергия активации. Активированный комплекс. Энергетическая диаграмма для экзотермических и эндотермических реакций.
 
\item 
Катализ. Механизмы реакций с участием катализатора.
 
\item 
Кинетика гетерогенных химических реакций. Основные стадии. Привести уравнение закона действующих масс для  гетерогенной химической реакции.

\item
Катализатор. Механизмх действия катализатора. Привести энергетическую диаграмму для одностадийного катализа.

\item
Гомогенный катализ. Основные положения. Привести схему. Чему равна скорость гомогенного катализа?

\item
Гетерогенный катализ. Основные стадии гетерогенного катализа. Каковы специфические особенности гетерогенного катализа? Как можно увеличить скорость гетерогенного катализа?

\end{enumerate}