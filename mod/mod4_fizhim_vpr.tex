%Вопросы к  модульному контролю. Физическая и коллоидная химия.
%Модуль 4
%
%Темы:
%1) Адсорбция. Поверхностные явления (вопросы 1-8)
%2) Дисперсные системы. Коллоидные растворы (вопросы 9-18)
%
%
\section{Адсорбция. Поверхностные явления}
%Адсорбция. Поверхностные явления.
% вопросы 1-8
\begin{enumerate}
\item
Избирательная адсорбция. Лиотропные ряды.
 
\item
Избирательная адсорбция. Правило Панета-Фаянса.
 
\item
Что такое ионообменная адсорбция? Иониты и их виды.
 
\item
Адсорбция электролитов. Механизмы образования двойного электрического слоя (ДЭС). Строение ДЭС. Влияние многозарядных ионов на строение ДЭС. Что такое сверхэквивалентная адсорбция?
 
\item
Полимолекулярная адсорбция. Теория БЭТ. ТеорияПоляни.
 
\item
Мономолекулярная адсорбция. Изотерма адсорбции.  Изотерма Лэнгмюра (график). Уравнение Генри (график). Дать объяснения.
 
\item
Уравнение Гиббса. Дать анализ уравнения.
 
\item
ПАВ и ПИВ. Их влияние на адсорбцию.
 
\item
Поверхностное натяжение. Дать определение. Полярность, дать определение. Правило Ребиндера. Зависимость поверхностного натяжения от полярности.
 
\end{enumerate}
%вопросы 1-8
%
\section{Дисперсные системы. Коллоидные растворы}
%Дисперсные системы. Коллоидные растворы.
%вопросы 9-18
\begin{enumerate}
\item
Дисперсные системы. Основные особенности. Классификация дисперсных систем. По агрегатному состоянию и по размеру частиц.
 
\item
Получение дисперсных систем методом конденсации. Привести уравнение радиуса зародыша новой фазы. Привести условия получения.
 
\item
Физическая конденсация. Основные методы. Дать описание.
 
\item
Агрегативная устойчивость дисперсных систем. Кинетические факторы устойчивости. Что такое седиментационная устойчивость коллоидных систем? Каковы основные условия этой устойчивости? Термодинамические факторы устойчивости.
 
\item
Теория устойчивости дисперсных систем. Потенциальная кривая зависимости сил взаимодействия между частицами. Природа сил отталкивания и притяжения.
 
\item
Получение коллоидных растворов. Основные условия их получения.
 
\item
Получение коллоидных растворов методом конденсации. Условия образования зародышей новой фазы, что такое степень пересыщения?
 
\item
Химическая конденсация, реакция окисления Привести реакцию и схему мицеллы.
 
\item
Химическая конденсация, реакция гидролиза, привести реакцию и схему мицеллы.
 
\item
Химическая конденсация, реакция двойного обмена. Привести реакцию и схему мицеллы.
 
\item
Получение коллоидных растворов методом диспергирования, теория Ребиндера. Эффект Ребиндера и его механизм.
 
\item
Физико-химическое диспергирование. Метод адсорбционной пептизации, привести пример и схему мицеллы.
 
\item
Физико-химическая пептизация. Метод промывания осадка. Привести пример реакции и схему мицеллы. Дать объяснение.
 
\item
Правило осадков Оствальда. Зависимость золя пептизированного осадка от концентрации электролита. Привести рисунок. Дать объяснение.

\item
Строение коллоидных частиц. Что такое мицелла (пример), интермицелярная жидкость. Что такое противоионы? Свободные и связанные противоионы (привести пример). Что такое адсорбционный и диффузионный слои, гранула (привести примеры). 
 
\item
Строение коллоидных частиц. Образование двойного электрического слоя (ДЭС) по Гельмгольцу (привести схему). Образование ДЭС по Штерну (привести схему).
 
\item
ДЭС. Дать понятие, привести схему. $\zeta$ -потенциал, дать объяснение. 
 
\item
Влияние одно-, двух-, трех-, четырехзарядных электролитов на строение ДЭС. Дать объяснение.
 
\item
Механизмы коагуляции золей: концентрационный и адсорбционный.
 
\item
Теория ДЛФО. Показать поведение коллоидных частиц в случае отсутствия коагуляции и когда наступает коагуляция. Дать объяснение (рисунок). Что такое потенциальный барьер коагуляции? Показать поведение коллоидных систем в случае образования структурированных систем (рисунок).
 
\item
Коагуляция. Коагуляция золей при действии электролита. Правило Шульце-Гарди. Что такое порог коагуляции? Коагулирующая способность. Влияние размера иона коагулятора на коагуляцию. 
 
\item
Влияние заряда иона коагулятора на коагуляцию индеферентного электролита. Неправильные ряды. (рисунок).
 
\item
Скорость коагуляции. Кинетика быстрой коагуляции Смолуховского (схема).
 
\item
Коагуляция золей смесями электролитов. Что такое аддитивность, антогонизм и синергизм электролитов. Их механизм. Привыкание золей. Положительное и отрицательное привыкание золей и их механизм (рисунок). 
 
\item
Защита коллоидных растворов от коагуляции. Механизм защитного действия. Солюбилизация.
 
\item
Электрокинетические явления. Электрофорез. Электроосмос. Потенциал седиментации. Потенциал течения.
\end{enumerate}
%вопросы 9-18