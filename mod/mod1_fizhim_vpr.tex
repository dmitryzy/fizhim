%Вопросы к  модульному контролю. Термодинамика
%вопрос 1
\section{Термодинамика}
\begin{enumerate}
\item 
Термодинамическая система. Типы термодинамических систем (открытая, закрытая и изолированная), (гомогенная, гетерогенная). Дайте определения. Приведите примеры.
 
\item 
Приведите формулировку нулевого закона термодинамики.
 
\item 
Первое начало термодинамики. Дать определение. Привести уравнение. Первое начало термодинамики для изобарного, изохорного и изотермического процессов. Назовите постоянные параметры в каждом процессе: изохорный процесс, изобарный процесс, изотермический процесс.
 
\item 
Теплоемкость. Определение теплоемкости в классической термодинамике. В каких единицах измеряется. Виды теплоемкостей: удельная, молярная, изобарная, изохорная.
 
\item 
Тепловой эффект химической реакции. Экзо- и эндотермические реакции. Укажите знаки теплового эффекта $Q$ и энтальпии  $\Delta_{r}H$ для экзотермической  и эндотермической реакций. Приведите примеры экзотермических и эндотермических реакций.
 
\item 
Зависимость теплового эффекта химической реакции от температуры. Уравнение Кирхгофа в интегральной и дифференциальной формах для изобарного процесса.
 
\item 
Зависимость теплового эффекта химической реакции от температуры. Уравнение Кирхгофа в интегральной и дифференциальной формах для изохорного процесса.
 
\item 
Что такое внутренняя энергия? Внутренняя энергия как термодинамическая функция. Изохорная теплоемкость. Привести уравнение.
 
\item 
Энтальпия как термодинамическая функция. В каких единицах измеряется энтальпия. Приведите определение изобарной теплоемкости. Привести уравнение.
 
\item 
Приведите определение стандартной энтальпии образования химических веществ. Чему равна стандартная энтальпия образования простых веществ? Назовите стандартные условия реакции (температура, давление).
 
\item 
Приведите определение стандартной энтальпии сгорания химических веществ. Чему равна стандартная энтальпия сгорания высших оксидов? Назовите стандартные условия реакции (температура, давление).
 
\item 
Экспериментальное определение теплового эффекта реакции (на примере реакции нейтрализации). Устройство и принцип работы калориметра. Что учитывает постоянная калориметра и как её определить?
 
%вопрос 2
\item 
Как по стандартной энтальпии образования исходных и конечных реагентов вычислить стандартную энтальпию химической реакции (описать метод рассчета)?
 
\item 
Как по стандартной энтальпии сгорания исходных и конечных реагентов вычислить стандартную энтальпию химической реакции (описать метод рассчета)?
 
\item 
Теплота растворения (определение). Из каких тепловых эффектов складывается  теплота  растворения твёрдого вещества?
 
\item 
Теплота гидратации (определение). Привести пример реакции гидратации.
 
\item 
Закон Гесса. Следствия из закона Гесса. Способы расчета энтальпий реакций с использованием закона Гесса (на конкретных примерах).
 
\item 
Уравнения состояния системы. Привести примеры уравнений состояния для идеального и реального газов. 
 
\item 
Термодинамические переменные. Экстенсивные и интенсивные переменные. Температура, давление, объем.
 
\item 
Температура. Единицы измерения температуры. Измерение температуры с помощью термометра. Устройство и принцип работы термометра.
 
\item 
Запись термохимических уравнений.
 
\item 
Второе начало термодинамики. Теорема Карно. Привести уравнение.
 
\item 
Энтропия как термодинамическая функция. Статистическая природа второго начала термодинамики. Изменение энтропии в различных фазовых превращениях (плавление, кристаллизация, испарение, конденсация).
 
\item 
Способ вычисления стандартной энтропии химической реакции по энтропиям образования исходных и конечных реагентов. Уточненный расчет энтропии для заданной температуры с использованием закона Кирхгофа.
 
%вопрос 3
\item 
Энергия Гельмгольца как термодинамическая функция. Критерий протекания самопроизвольных процессов в изохорно-изотермических условиях.
 
\item 
Энергия Гиббса как термодинамическая функция. Критерий протекания самопроизвольных процессов в изобарно-изотермических условиях.
 
\item 
Химический потенциал как термодинамическая функция. Выражение химического потенциала через энергию Гиббса, энтальпию,  энергию Гельмгольца и энтропию.
 
\item 
Изотерма химической реакции. Записать уравнение.
 
\item 
Изобара химической реакции. Изохора химической реакции. Записать уравнения.
 
\item 
Запись выражений константы равновесия химических реакций.
 
\item 
Взаимосвязь между константами равновесия $K_{x}$, $K_{C}$, $K_{P}$.
 
\item 
Признаки и условия химического равновесия. Методы расчета равновесного состава для газовых систем.
 
\item 
Сдвиг химического равновесия. Принцип Ле-Шателье. Как влияет на смещение равновесия изменение давления, температуры, концентраций реагирующих веществ.
 
\item 
Условия фазового равновесия. Правило фаз Гиббса. Степень свободы, компонент, фаза.
 
\item 
Фазовое равновесие. Уравнение Клаузиуса-Клапейрона. Фазовые переходы. Виды фазовых переходов. 
 
\item 
Дать определение гомогенной и гетерогенной системам. 
Экстракционное равновесие. Закон распределения Нернста-Шилова. Дать формулировку, записать уравнение. Что такое коэффициент распределения и от чего он зависит? Что такое коэффициент распределения и от чего он зависит?
 
\end{enumerate}