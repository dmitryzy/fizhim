%Вопросы к  модульному контролю. Физическая химия.
%Модуль 3
%
%Растворы электролитов. Электрохимия 
\section{Растворы. Электрохимия}
\begin{enumerate}
\item 
Равновесие жидкость - пар в двухкомпонентных системах. Фазовые диаграммы двукомпонентных систем. Азеотропные смеси. Законы Коновалова.
\item
Идеальные растворы.  Закон Рауля. Зависимость общего и парциального давления от состава раствора. Закон Дальтона.

\item 
Дать определение идеального раствора. Привести выражение химического потенциала для идеального раствора. Повышение температуры кипения  и понижение температуры замерзания растворов.

\item
Реальные растворы.  Закон Генри. Активность. Привести выражение химического потенциала для реального раствора. Уравнения Гиббса-Дюгема-Маргулеса. Обобщенное уравнение Гиббса-Дюгема.

\item
Что такое электролитическая диссоциация? Виды диссоциации. Степень диссоциации, константа диссоциации и их связь.
Сильные и слабые электролиты.  
\item
Сильные  электролиты. Теория сильных электролитов Дебая-Хюккеля. Основные положения.  
\item
Сильные электролиты. Электрофоретический эффект торможения. Релаксационный эффект. Дать понятие, привести схемы.
 
\item
Тормозящие эффекты в сильных электролитах. Что такое ионная сила электролита, ионнная атмосфера (привести рисунок).  Что такое активность, подвижность ионов. Средний коэффициент активности электролита.
 
\item
Подвижность ионов. Закон Кольрауша. Числа переноса ионов.
 
\item
В чем причины высокой подвижности ионов $H_{3}O^{+}$ и $OH^{-}$? Объясните механизм перемещения этих ионов в растворе электролита.
 
\item
Что такое изотонический коэффициент? Его влияние на законы идеальных растворов.
 
\item
Удельная электропроводность. Определение. В каких единицах измеряется? Зависимость удельной электропроводности от концентрации для сильных и слабых электролитов (привести схемы).
 
\item
Эквивалентная электропроводность и ее зависимость от концентрации. Определение. В каких единицах измеряется? Что такое разведение электролита, предельная эквивалентная электропроводность?
 
\item
Зависимость электропроводности от температуры.
 
\item
Возникновение двойного электрического слоя (ДЭС) на границе металл-раствор (когда $\mu_{Me}<\mu_{s}$ и $\mu_{Me}<\mu_{s}$). Образование двойного электрического слоя (ДЭС) по Гельмгольцу (привести схему). Образование ДЭС по Штерну (привести схему).
 
\item
Гальванический элемент. Устройство и схема гальванического элемента. Токообразующая реакция. 
 
\item
Термодинамические характеристики гальванического элемента. Температурный коэффициент ЭДС.
 
\item
Уравнение Нернста для медно-цинкового гальванического элемента, для водородного электрода, для хлор-серебрянного электрода. Привести реакции.
 
\item
Классификации электрокинетических явлений. Электроосмос. Потенциал оседания. Потенциал течения.
 
\item
Электрофорез. Эффекты, осложняющие электрофорез. 
 
\item
Электролиз. Законы Фарадея. Применение электролиза.
 
\item
Электролиз. Поляризация. Причины поляризации. Кинетика электрохимических процессов.
 
\item
Электроды. Классификация электродов. 
 
\item
Электродные процессы в электролитах. Электродный потенциал. Возникновение электродного потенциала на границе раздела фаз. Механизм возникновения электродного потенциала.
 
\item
Закон Оствальда. Вывод уравнения для одноосновных кислот. Записать уравнения закона Оствальда для данного электролита через степень диссоциации и через эквивалентную электропроводность.
 
\item
Закон Оствальда. Вывод уравнения для двухосновных кислот. Записать уравнения закона Оствальда для данного электролита через степень диссоциации и через эквивалентную электропроводность.
\end{enumerate}