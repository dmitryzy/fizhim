%Вопросы к  модульному контролю. 
%Физическая химия
%Модуль: Химическая кинетика.
%
%Общий список
%
%%1
\section{Химическая кинетика}
\begin{enumerate}
\item 
Основные понятия и постулаты формальной кинетики. Прямая и обратная кинетические задачи. Параметры кинетических уравнений. Константа скорости реакции.
 
\item 
Дать определение скорости химических реакций в  гомогенных системах. В каких единицах измеряется скорость химической реакции? Записать формулу для расчета скорости с пояснениями. От каких факторов зависит скорость гомогенных  процессов? 
 
\item
Скорость в гетерогенной реакции. Привести уравнение скорости гетерогенных химических реакций. В каких единицах измеряется скорость гетерогенной химической реакции? От каких факторов зависит скорость гетерогенных процессов? 
 
\item 
Влияние концентрации на скорость реакции. Закон действующих масс. Запись уравнения закона действующих масс для прямой и обратной реакций (через парциальные давления реагентов и молярные концентрации). Уравнение закона действующих масс для обратимой реакции.
 
\item
Что такое порядок и молекулярность реакции? Причины несовпадения порядка и молекулярности реакций. Время полупревращения.
 
\item
Способы экспериментального определения порядка реакции (интегральные методы).
 
\item 
Способы экспериментального определения порядка реакции (дифференциальный метод).
 
\item
Реакции нулевого порядка. Записать выражение закона действующих масс для реакции 0-го порядка. 
 
\item
Реакции первого порядка. Записать выражение закона действующих масс для реакции 1-го порядка.
 
\item 
Реакции второго порядка. Записать выражение закона действующих масс для реакции 2-го порядка. 
 
\item
Реакции третьего порядка. Записать выражение закона действующих масс для реакции 3-го порядка.
 
\item
Дать определение обратимой реакции. Почему обратимые реакции носят динамический характер? Запишите соотношение, связывающее константу равновесия обратимой реакции и кинетические константы прямой и обратной реакций.
 
%%2
\item
Дать определение энергии активации. Что такое активированный комплекс? Записать уравнение, связывающее энергии активации  прямой и обратной реакций.
 
\item 
Уравнение Аррениуса. Параметры уравнения Аррениуса (энергия активации, стерический множитель). Способы определения опытной энергии активации.
 
\item
Правило Вант-Гоффа. Что такое температурный коэффициент реакции? 
 
\item 
Влияние температуры на скорость химической реакции. Привести энергетическую диаграмму для экзотермической реакции.
Привести энергетическую диаграмму для эндотермической реакции.
 
\item
Основные положения теории активированного комплекса.
 
\item
Основные положения теории активных соударений.
 
\item
Кинетика гетерогенных процессов. Теории гетерогенного катализа.
 
\item
Катализ. Мультиплетная теория. Теория активных ансамблей. Электронная теория.
 
%
%%3
\item
Что называется элементарной стадией химической реакции? По какому признаку классифицируют элементарные реакции?
 
\item
Классификация сложных реакций. Последовательные реакции. Каковы особенности кинетики последовательных реакций? Что такое лимитирующая стадия?
 
\item
Классификация сложных реакций. Параллельные реакции. Каковы особенности кинетики параллельных реакций? 
 
\item 
Скорости реакций в открытых системах. Реактор идеального смешения и реактор идеального вытеснения. Уравнение для стационарной скорости реакции в реакторах идеального смешения и идеального вытеснения.
 
\item
Гетерогенные реакции. Диффузия. Условия протекания реакции в диффузионном, кинетическом и переходном режимах.
 
\item
Цепные реакции. Механизм цепных реакций. Какие этапы характерны для цепных реакций?
 
\item
Фотохимические реакции. Основные закономерности протекания фотохимических реакций. Что такое квантовый выход реакции?
 
\item
Что такое катализатор. Виды катализа. Привести общие закономерности катализа.
 
\item
Гомогенный катализ. Привести схему гомогенного катализа.
 
\item
Гетерогенный катализ? Основные стадии гетерогенного катализа.
\end{enumerate}