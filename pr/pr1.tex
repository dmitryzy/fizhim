\chapter{Практическое занятие 1. Термохимия}
\section{Задачи для самостоятельного решения}
\begin{Task}
Один моль идеального газа, взятого при 25 $^{o}$C и 100 атм, расширяется обратимо и изотермически до 5 атм. Рассчитайте работу, поглощенную теплоту,  $\Delta U$ и  $\Delta H$.
\end{Task}
\begin{Task}
Рассчитайте изменение энтальпии кислорода (идеальный газ) при изобарном расширении от 80 до 200 л при нормальном атмосферном давлении.
\end{Task}
\begin{Task}
Рассчитайте количество теплоты, необходимое для нагревания воздуха в квартире общим объемом 600 м$^{3}$ от 20 $^{o}$С до 25 $^{o}$С. Примите, что воздух - это идеальный двухатомный газ, а давление при исходной температуре нормальное. Найдите  $\Delta U$ и $\Delta H$ для процесса нагревания воздуха.
\end{Task}
\begin{Task}
Человеческий организм в среднем выделяет 104 кДж в день благодаря метаболическим процессам. Основной механизм потери этой энергии - испарение воды. Какую массу воды должен ежедневно испарять организм для поддержания постоянной температуры? Удельная теплота испарения воды 2260 Дж/г. На сколько градусов повысилась бы температура тела, если бы организм был изолированной системой? Примите, что средняя масса человека - 65 кг, а теплоемкость равна теплоемкости жидкой воды.
\end{Task}
\begin{Task}
Один моль метана, взятый при 25 $^{o}$С и 1 атм, нагрет при постоянном давлении до удвоения объема. Мольная теплоемкость метана дается выражением:
$C_{p} = 5,34 + 0,0115\cdot T$ кал/(мольК).
Рассчитайте  $\Delta U$ и  $\Delta H$ для этого процесса. Метан можно считать идеальным газом.
\end{Task}
\begin{Task}
Сколько тепла потребуется на перевод 500 г Al ($T_{\textsc{пл}}=658^{o}$С, $\Delta H_{\textsc{пл}} = 92,4$ кал/г), взятого при комнатной температуре, в расплавленное состояние, если $C_{p}(Al) = 0,183 + 1,096\cdot 10^{-4}T$ кал/(г К)?
\end{Task}
\begin{Task}
Стандартная энтальпия реакции 
$CaCO_{3}(\textsc{тв}) = CaO(\textsc{тв}) + CO_{2}(\textsc{г})$, протекающей в открытом сосуде при температуре 1000 К, равна 169 кДж/моль. Чему равна теплота этой реакции, протекающей при той же температуре, но в закрытом сосуде?
\end{Task}
\begin{Task}
Рассчитайте энтальпию образования $N_{2}O_{5}(\textsc{г})$ при $T = 298$ К на основании следующих данных:

$2NO(\textsc{г}) + O_{2}(\textsc{г}) = 2NO_{2}(\textsc{г})$, $\Delta H_{1}^{0} = -114,2$ кДж/моль,

$4NO_{2}(\textsc{г}) + O_{2}(\textsc{г}) = 2N_{2}O_{5}(\textsc{г})$, $\Delta H_{2}^{0} = -110,2$ кДж/моль,

$N_{2}(\textsc{г}) + O_{2}(\textsc{г}) = 2NO(\textsc{г}),  \Delta H_{3}^{0} = 182,6$ кДж/моль.
\end{Task}
\begin{Task}
Энтальпии сгорания -глюкозы, -фруктозы и сахарозы при 25 $^{o}$С равны -2802, -2810 и -5644 кДж/моль, соответственно. Рассчитайте теплоту гидролиза сахарозы.
\end{Task}
\begin{Task}
Определите энтальпию образования диборана $B_{2}H_{6}(\textsc{г})$ при $T = 298$ К из следующих данных:

$B_{2}H_{6}(\textsc{г}) + 3O_{2}(\textsc{г}) = B_{2}O_{3}(\textsc{тв}) + 3H_{2}O(\textsc{г})$, $\Delta H_{1}^{0} = -2035,6$ кДж/моль,

$2B(\textsc{тв}) + 3/2 O2(\textsc{г}) = B2O3(\textsc{тв})$, $\Delta H_{2}^{0} = -1273,5$ кДж/моль,

$H_{2}(\textsc{г}) + 1/2 O_{2}(\textsc{г}) = H_{2}O(\textsc{г})$, $\Delta H_{3}^{0} = -241,8$ кДж/моль.
\end{Task}
\begin{Task}
Рассчитайте теплоту образования сульфата цинка из простых веществ при $T = 298$ К на основании следующих данных:

$ZnS = Zn + S$, $\Delta H_{1}^{0} = 200,5$ кДж/моль,

$2ZnS + 3O_{2} = 2ZnO + 2SO_{2}$, $\Delta H_{2}^{0} = -893,5$ кДж/моль,

$2SO_{2} + O_{2} = 2SO_{3}$, $\Delta H_{3}^{0} = -198,2$ кДж/моль,

$ZnSO_{4} = ZnO + SO_{3}$, $\Delta H_{4}^{0} = 235,0$ кДж/моль.
\end{Task}
\begin{Task}
Найдите $\Delta_{r}H_{298}^{0}$ для реакции
$$CH_{4} + Cl_{2} = CH_{3}Cl + HCl$$
если известны теплоты сгорания метана (-890,6 кДж/моль), хлорметана (-689,8 кДж/моль), водорода ( -285,8 кДж/моль) и теплота образования $HCl$ (-92,3 кДж/моль)).
\end{Task}
\begin{Task}
Рассчитайте стандартный тепловой эффект реакции
$$CaSO_{4}(\textsc{тв}) + Na_{2}CO_{3}(aq) = CaCO_{3}(\textsc{тв}) + Na_{2}SO_{4}(aq)$$   при 298 К.
\end{Task}
\begin{Task}
Зависимость теплового эффекта реакции
$$CH_{3}OH + 3/2O_{2} =CO_{2} + 2H_{2}O$$ от температуры выражается уравнением:
$$\Delta H=-684,7\cdot 10^{3}+36,77T-38,56\cdot 10^{-3}T^{2}+8,21\cdot 10^{-6}T^{3}+2,88\cdot10^{5}T^{-1}$$
Рассчитайте изменение теплоемкости  $\Delta C_{p}$ для этой реакции при 500 К.
\end{Task}
\begin{Task}
Энтальпия диссоциации карбоната кальция при 900 $^{o}$С и давлении 1 атм равна 178 кДж/моль. Выведите уравнение зависимости энтальпии реакции от температуры и рассчитайте количество теплоты, поглощенное при разложении 1 кг карбоната кальция при 1000 $^{o}$С и 1 атм, если даны мольные теплоемкости (в Дж/(моль. К)):

$Cp(CaCO_{3}(\textsc{тв})) = 104,5 + 21,92\cdot 10^{-3}T - 25,94\cdot 10^{5}T^{-2}$,

$Cp(CaO(\textsc{тв})) = 49,63 + 4,52\cdot 10^{-3}T - 6,95\cdot 10^{5}T^{-2}$,

$Cp(CO_{2}(\textsc{г})) = 44,14 + 9,04\cdot 10^{-3}T - 8,53\cdot 10^{5}T^{-2}$.
\end{Task}
\begin{Task}
Рассчитайте изменение энтропии при нагревании 0,4 моль хлорида натрия от 20 до 850 $^{o}$С. Мольная теплоемкость хлорида натрия равна:

$C_{p}(NaCl(\textsc{тв})) = 45,94 + 16,32\cdot 10^{-3} T$ Дж/(мольК),

$C_{p}(NaCl(\textsc{ж})) = 66,53$ Дж/(мольК).
Температура плавления хлорида натрия 800 $^{o}$С, теплота плавления 31,0 кДж/моль.
\end{Task}