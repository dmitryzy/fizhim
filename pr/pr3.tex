%Химическая кинетика
\chapter{Практическое занятие 3. Химическая кинетика}
\section{Задачи для самостоятельного решения}
\begin{Task}
Чему равен порядок элементарных реакций? Записать для них уравнения закона действующих масс.
$Cl + H_{2} = HCl + H$; $2NO + Cl_{2} = 2NOCl$.
\end{Task}
\begin{Task}
Какие из перечисленных величин могут принимать отрицательные или дробные значения: скорость реакции, порядок реакции, молекулярность реакции, константа скорости, стехиометрический коэффициент?
\end{Task}
\begin{Task}
 Во сколько раз увеличится скорость газофазной элементарной реакции $A = 2D$ при увеличении общего давления в 3 раза?
\end{Task}
\begin{Task}
Определите порядок реакции, если размерность константы скорости: л$^{2}$/(моль$^{2}$с).
\end{Task}
\begin{Task}
Константа скорости газовой реакции 2-го порядка при 25 $^{o}$С равна $10^{3}$~л/~(~моль~с~). Чему равна эта константа, если кинетическое уравнение выражено через давление в атмосферах? 
\end{Task}
\begin{Task}
Для газофазной реакции n-го порядка $nA\rightarrow B$ выразите скорость образования $B$ через суммарное давление.
\end{Task}
\begin{Task}
Константы скорости прямой и обратной реакции равны 2,2~л~/~(~моль с) и 3,8 л/(моль с). По какому из перечисленных ниже механизмов могут протекать эти реакции: а) $A + B = D$; б) $A+ B = 2D$; в) $A = B + D$; г) $2A = B$. 
\end{Task}
\begin{Task}
Скорость реакции 2-го порядка $A + B\rightarrow D$ равна $2,7\cdot 10^{-7}$ моль/(л. с) при концентрациях веществ $A$ и $B$, соответственно, $3,0\cdot 10^{-3}$ моль/л и $2,0$ моль/л. Рассчитайте константу скорости. 
\end{Task}
\begin{Task}
В реакции 2-го порядка $A + B\rightarrow 2D$ начальные концентрации веществ $A$ и $B$ равны по 1,5 моль/л. Скорость реакции равна $2,0\cdot 10^{-4}$ моль/(л с) при $C_{A} = 1,0$ моль/л. Рассчитайте константу скорости и скорость реакции при [B] = 0,2 моль/л.  
\end{Task}
\begin{Task}
В реакции 2-го порядка $A + B\rightarrow 2D$ начальные концентрации веществ $A$ и $B$ равны, соответственно, 0,5 и 2,5 моль/л. Во сколько раз скорость реакции при $C_{A}= 0,1$ моль/л меньше начальной скорости?  
\end{Task}
\begin{Task}
Скорость газофазной реакции описывается уравнением 
$r = kC_{A}^{2}C_{B}$.
При каком соотношении между концентрациями $A$ и $B$ начальная скорость реакции будет максимальна при фиксированном суммарном давлении?  
\end{Task}
\begin{Task}
Реакция первого порядка протекает на 30\% за 7 мин. Через какое время реакция завершится на 99\%?
\end{Task}
\begin{Task}
Период полураспада радиоактивного изотопа $^{90}Sr$, который попадает в атмосферу при ядерных испытаниях, -- 28,1 лет. Предположим, что организм новорожденного ребенка поглотил 1,00 мг этого изотопа. Сколько стронция останется в организме через а) 18 лет, б) 70 лет, если считать, что он не выводится из организма? 
\end{Task}
\begin{Task}
Константа скорости для реакции первого порядка 
$SO_{2}Cl_{2} = SO_{2} + Cl_{2}$
 равна $2,2\cdot 10^{-5}$ с$^{-1}$ при 320~$^{o}$~С. Какой процент $SO_{2}Cl_{2}$ разложится при выдерживании его в течение 2 ч при этой температуре?
\end{Task}
\begin{Task}
Реакция второго порядка $2A\rightarrow B$ протекает в газовой фазе. Начальное давление равно $P_{0}$ ($B$ отсутствует). Найдите зависимость общего давления от времени. Через какое время давление уменьшится в 1,5 раза по сравнению с первоначальным? Какова степень протекания реакции к этому времени?  
\end{Task}
\begin{Task}
Вещество A смешали с веществами B и C в равных концентрациях 1 моль/л. Через 1000 с осталось 50\% вещества А. Сколько вещества А останется через 2000 с, если реакция имеет: а) нулевой, б) первый, в) второй, в) третий общий порядок?
\end{Task}
\begin{Task}
Какая из реакций - первого, второго или третьего порядка - закончится быстрее, если начальные концентрации веществ равны 1 моль/л и все константы скорости, выраженные через моль/л и с, равны 1?
\end{Task}
\begin{Task}
При определенной температуре 0,01 М раствор этилацетата омыляется 0,002 М раствором NaOH на 10\% за 23 мин. Через сколько минут он будет омылен до такой же степени 0,005 М раствором KOH? Считайте, что данная реакция имеет второй порядок, а щелочи диссоциированы полностью.
\end{Task}
\begin{Task}
Реакция второго порядка $A + B\rightarrow P$ проводится в растворе с начальными концентрациями $C_{A0} = 0,050$ моль/л и $C_{B0} = 0,080$ моль/л. Через 1 ч концентрация вещества А уменьшилась до 0,020 моль/л. Рассчитайте константу скорости и периоды полураспада обоих веществ. 
\end{Task}
\begin{Task}
В газофазной реакции $A + B\rightarrow P$ скорость измерялась при различных парциальных давлениях реагентов (температура 300 К). Определите порядки реакции по веществам А и В. Исходные данные:\\
\begin{tabular}{|c|c|c|c|}
\hline 
\No опыта & $p_{A}$, мм рт. ст. & $p_{B}$, мм рт. ст. &$r$, моль/(л с) \\
\hline
1 & 4,0 & 15,0 & $2,59\cdot 10^{-7}$ \\
\hline
2 & 9,0 & 12,0 &  $1,05\cdot  10^{-6}$\\
\hline
3 & 13,0 & 9,0 & $1,64\cdot 10^{-6}$\\
\hline
\end{tabular} 
\end{Task}
\begin{Task}
Вычислите, при какой температуре реакция закончится через 15 мин, если при 20 $^o$С на это требуется 2 ч. Температурный коэффициент скорости равен 3.
\end{Task}
\begin{Task}
Какой должна быть энергия активации, чтобы скорость реакции увеличивалась в 3 раза при возрастании температуры на 10  $^o$С а) при 300 К; б) при 1000 К?
\end{Task}
\begin{Task}
Энергия активации некоторой реакции в 1,5 раза больше, чем энергия активации другой реакции. При нагревании от $T_{1}$ до $T_{2}$ константа скорости второй реакции увеличилась в a раз. Во сколько раз увеличилась константа скорости первой реакции при нагревании от $T_{1}$ до $T_{2}$?
\end{Task}
\begin{Task}
Реакция первого порядка имеет энергию активации 104,5 кДж/моль и предэкспоненциальный множитель $5\cdot 10^{13}$ с$^{-1}$. При какой температуре время полураспада для данной реакции составит: а) 1 мин; б) 30 дней?
\end{Task}
\begin{Task}
В необратимой реакции 1-го порядка за 20 мин при 125 $^o$С степень превращения исходного вещества составила 60\%, а при 145 $^o$C такая же степень превращения была достигнута за 5,5 мин. Найдите константы скорости и энергию активации данной реакции .
\end{Task}
\begin{Task}
Реакция 1-го порядка при температуре 25 оС завершается на 70\% за 15 мин. При какой температуре реакция завершится на 50\% за 15 мин, если энергия активации равна 50 кДж/моль?
\end{Task}
%
%\begin{Task}
%Реакция разложения $2HI\rightarrow H_{2} + I_{2}$ имеет 2-й порядок с константой скорости  $k = 5,95\cdot 10^{-6}$ л/(моль. с). Вычислите скорость реакции при давлении 1 атм, мольной доле исходного реагента 0,001 и температуре 600 К. 
%\end{Task}
%\begin{Task}
%Время полураспада вещества при 323 К равно 100 мин, а при 353 К - 15 мин. Определите температурный коэффициент скорости.
%\end{Task}