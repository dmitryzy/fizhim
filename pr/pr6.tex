%Коллигативные свойства растворов
\chapter{ Практическое занятие 6. Коллигативные свойства растворов}
\section{Задачи для самостоятельного решения}
\begin{Task}
Рассчитать минимальную осмотическую работу, совершаемую почками для выделения мочевины при 36,6$^{o}$ C, если концентрация мочевины в плазме 0,005 моль/л, а в моче 0,333 моль/л.
\end{Task}
\begin{Task}
10 г полистирола растворено в 1 л бензола. Высота столбика раствора (плотностью 0,88 г/см$^{3}$) в осмометре при 25$^{o}$C равна 11,6 см. Рассчитать молярную массу полистирола.
\end{Task}
\begin{Task}
Белок сывороточный альбумин человека имеет молярную массу 69 кг/моль. Рассчитать осмотическое давление раствора 2 г белка в 100 см$^{3}$ воды при 25$^{o}$C в Па и в мм столбика раствора. Считать плотность раствора равной 1,0 г/см$^{3}$.
\end{Task}
\begin{Task}
При 30$^{o}$C давление пара водного раствора сахарозы равно 31,207 мм рт. ст. Давление пара чистой воды при 30$^{o}$C равно 31,824 мм рт. ст. Плотность раствора равна 0,99564 г/см$^{3}$. Чему равно осмотическое давление этого раствора?
\end{Task}
\begin{Task}
Плазма человеческой крови замерзает при $-0,56^{o}$C. Каково ее осмотическое давление при 37$^{o}$C, измеренное с помощью мембраны, проницаемой только для воды?
\end{Task}
\begin{Task}
Молярную массу фермента определяли, растворяя его в воде и измеряя высоту столбика раствора в осмометре при 20$^{o}$C, а затем экстраполируя данные к нулевой концентрации. Получены следующие данные:\\
\begin{tabular}{|c|c|c|c|c|}
\hline 
$C$, мг/см$^{3}$ & 3,211 & 4,618 & 5,112 & 6,722\\
\hline
$h$, см & 5,746 & 8,238 & 9,119 & 11,990\\
\hline
\end{tabular} 

Рассчитать молярную массу фермента.
\end{Task}
\begin{Task}
Молярную массу липида определяют по повышению температуры кипения. Липид можно растворить в метаноле или в хлороформе. Температура кипения метанола 64,7$^{o}$C, теплота испарения 262,8 кал/ г. Температура кипения хлороформа 61,5$^{o}$C, теплота испарения 59,0 кал/г. Рассчитайте эбулиоскопические постоянные метанола и хлороформа. Какой растворитель лучше использовать, чтобы определить молярную массу с максимальной точностью?
\end{Task}
\begin{Task}
Рассчитать температуру замерзания водного раствора, содержащего 50,0 г этилен-гликоля в 500 г воды.
\end{Task}
\begin{Task}
Раствор, содержащий 0,217 г серы и 19,18 г $CS_{2}$, кипит при 319,304 К. Температура кипения чистого $CS_{2}$ равна 319,2 К. Эбулиоскопическая постоянная $CS_{2}$ равна 2,37 К кг/моль. Сколько атомов серы содержится в молекуле серы, растворенной в $CS_{2}$?
\end{Task}
\begin{Task}
68,4 г сахарозы растворено в 1000 г воды. Рассчитать: а) давление пара, б) осмотическое давление, в) температуру замерзания, г) температуру кипения раствора. Давление пара чистой воды при 20$^{o}$C равно 2314,9 Па. Криоскопическая и эбулиоскопическая постоянные воды равны 1,86 и 0,52 К кг/моль соответственно.
\end{Task}
\begin{Task}
Раствор, содержащий 0,81 г углеводорода H(CH2)nH и 190 г бромистого этила, замерзает при 9,47$^{o}$C. Температура замерзания бромистого этила 10,00$^{o}$C, криоскопическая постоянная 12,5 К. кг/моль. Рассчитать n.
\end{Task}
\begin{Task}
При растворении 1,4511 г дихлоруксусной кислоты в 56,87 г четыреххлористого углерода точка кипения повышается на 0,518 град. Температура кипения $CCl_{4}$ 76,75$^{o}$C, теплота испарения 46,5 кал/г. Какова кажущаяся молярная масса кислоты? Чем объясняется расхождение с истинной молярной массой?
\end{Task}
\begin{Task}
Некоторое количество вещества, растворенное в 100 г бензола, понижает точку его замерзания на 1,28$^{o}$C. То же количество вещества, растворенное в 100 г воды, понижает точку ее замерзания на 1,395$^{o}$C. Вещество имеет в бензоле нормальную молярную массу, а в воде полностью диссоциировано. На сколько ионов вещество диссоциирует в водном растворе? Криоскопические постоянные для бензола и воды равны 5,12 и 1,86 К кг/моль.
\end{Task}
\begin{Task}
Рассчитать идеальную растворимость антрацена в бензоле при 25$^{o}$C в единицах моляльности. Энтальпия плавления антрацена при температуре плавления (217$^{o}$C) равна 28,8 кДж/моль.
\end{Task}
\begin{Task}
Рассчитать растворимость п-дибромбензола в бензоле при 20 и 40$^{o}$C, считая, что образуется идеальный раствор. Энтальпия плавления n-дибромбензола при температуре его плавления (86,9$^{o}$C) равна 13,22 кДж/моль.
\end{Task}
\begin{Task}
Рассчитать растворимость нафталина в бензоле при 25$^{o}$C, считая, что образуется идеальный раствор. Энтальпия плавления нафталина при температуре его плавления (80,0$^{o}$C) равна 19,29 кДж/моль.
\end{Task}
\begin{Task}
Рассчитать растворимость антрацена в толуоле при 25$^{o}$C, считая, что образуется идеальный раствор. Энтальпия плавления антрацена при температуре плавления (217$^{o}$C) равна 28,8 кДж/моль.
\end{Task}
\begin{Task}
Рассчитать температуру, при которой чистый кадмий находится в равновесии с раствором Cd – Bi, мольная доля Cd в котором равна 0,846. Энтальпия плавления кадмия при температуре плавления (321,1$^{o}$C) равна 6,23 кДж/моль.
\end{Task}