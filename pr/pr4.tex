%Электрохимия
\chapter{ Практическое занятие 4. Электрохимия}
\section{Задачи для самостоятельного решения}
\begin{Task}
В гальваническом элементе при температуре 298 К обратимо протекает реакция $Cd + 2AgCl = CdCl_{2} + 2Ag$. Рассчитать изменение энтропии реакции, если стандартная ЭДС элемента $E_{0}= 0,6753$В, а стандартные энтальпии образования $CdCl_{2}$ и $AgCl$ равны $-389,7$ и $-126,9$ кДж/моль соответственно. 
\end{Task}
\begin{Task}
ЭДС элемента, в котором обратимо протекает реакция $$0,5Hg_{2}Cl_{2} + Ag = AgCl + Hg$$, равна 0,456 В при 298 К и 0,439 В при 293 К. Рассчитать $\Delta G$,  $\Delta H$ и  $\Delta S$ реакции.
\end{Task}
\begin{Task}
Вычислить тепловой эффект реакции $Zn + 2AgCl = ZnCl_{2} + 2Ag$, протекающей в гальваническом элементе при 273 К, если ЭДС элемента $E=1,015$В и температурный коэффициент ЭДС равен $-4,02\cdot 10^{-4}$ В/K.
\end{Task}
\begin{Task}
Рассчитать стандартный электродный потенциал пары $Fe^{3+}/Fe$ по данным таблицы стандартных электродных потенциалов для пар $Fe^{2+}/Fe$ и $Fe^{3+}/Fe^{2+}$.
\end{Task}
\begin{Task}
Рассчитать константу равновесия реакции диспропорционирования $$2Cu^{+}=Cu^{2+}+Cu$$ при 25$^{o}$C.
\end{Task}
\begin{Task}
Рассчитать константу равновесия реакции $ZnSO_{4} + Cd = CdSO_{4} + Zn$ при 25$^{o}$C по данным о стандартных электродных потенциалах.
\end{Task}
\begin{Task}
ЭДС элемента $Pt\mid H_{2}\mid HCl \parallel AgCl \mid Ag$ при 25$^{o}$C равна 0,322 В. Чему равен $pH$ раствора $HCl$.
\end{Task}
\begin{Task}
Растворимость $Cu_{3}(PO_{4})_{2}$ в воде при 25$^{o}$C равна $1,6\cdot 10^{-8}$моль/кг. Рассчитать ЭДС элемента 
\end{Task}
\begin{Task}
Раствор NaNO3 имеет ионную силу 0.30 моль. кг-1. Чему равна моляльность раствора Al2(SO4)3. имеющего такую же ионную силу.
\end{Task}
\begin{Task}
Рассчитать моляльность раствора $Al(NO_{3})_{3}$, имеющего ионную силу 0,30 моль/кг.
\end{Task}
\begin{Task}
Рассчитать ионную силу раствора, содержащего 0,10 моль/кг $KCl$ и 0,20 моль/кг $CuSO_{4}$.
\end{Task}
\begin{Task}
Средний ионный коэффициент активности 0,1 M водного раствора $H_{2}SO_{4}$ при 25$^{o}$C равен 0,265. Рассчитать активность $H_{2}SO_{4}$ в растворе.
\end{Task}
\begin{Task}
Средний ионный коэффициент активности 0,1 M водного раствора $HCl$ при 25$^{o}$C равен 0,796. Рассчитать активность $HCl$ в этом растворе.
\end{Task}
\begin{Task}
Водные растворы сахарозы и $KNO_{3}$ изотоничны при концентрациях 1,00 и 0,60 моль/л соответственно. Найти кажущуюся степень диссоциации $KNO_{3}$ в растворе.
\end{Task}
\begin{Task}
Осмотическое давление крови составляет 0,811 МПа. Какова должна быть концентрация раствора $NaCl$, чтобы он был изотоничен с кровью.  Принять степень диссоциации $NaCl$ равной 0,950.
\end{Task}
\begin{Task}
Водный раствор, содержащий 0,225 моль/кг $NaOH$, замерзает при $-0,667^{o}$C. Определить кажущуюся степень диссоциации $NaOH$ в этом растворе, если криоскопическая константа воды равна 1,86.
\end{Task}
\begin{Task}
Эквивалентная электропроводность раствора гидроксида этиламмония 

$C_{2}H_{5}NH_{3}OH$ при бесконечном разведении равна 232,6 См см$^{2}$/моль. Рассчитать константу диссоциации гидроксида этиламмония, эквивалентную электропроводность раствора, степень диссоциации и концентрацию ионов гидроксила в растворе при разведении 16 л/моль. если удельная электропроводность раствора при данном разведении равна $1,312\cdot 10^{-3}$ См/см.
\end{Task}
\begin{Task}
Константа диссоциации масляной кислоты $C_{3}H_{7}COOH$ равна $1,74\cdot 10^{-5}$ моль/л. Эквивалентная электропроводность раствора при разведении 1024 л/моль равна 41,3 См см$^{2}$/моль. Рассчитать степень диссоциации кислоты и концентрацию ионов водорода в этом растворе, а также эквивалентную электропроводность раствора при бесконечном разведении.
\end{Task}
\begin{Task}
Эквивалентная электропроводность $1,59\cdot 10^{-4}$ моль/л раствора уксусной кислоты при 25$^{o}$C равна 12,77 См см$^{2}$/моль. Рассчитать константу диссоциации кислоты и $pH$ раствора.
\end{Task}
\begin{Task}
Константа диссоциации гидроксида аммония равна $1,79\cdot 10^{-5}$ моль/л. Рассчитать концентрацию $NH_{4}OH$, при которой степень диссоциации равна 0,01. и эквивалентную электропроводность раствора при этой концентрации. 
\end{Task}
\begin{Task}
Рассчитать удельную электропроводность $1,0\cdot 10^{-3}$ M водного раствора $NaCl$ при 25$^{o}$C, считая, что подвижности ионов при этой концентрации равны их предельным подвижностям. Через слой раствора длиной 1 см, заключенный между электродами площадью 1 см$^{2}$. пропускают ток силой 1 мА. Какое расстояние пройдут ионы $Na^{+}$ и $Cl^{-}$ за 10 минут? 
\end{Task}
\begin{Task}
Удельная электропроводность водного раствора сильного электролита при 25$^{o}$C равна 109,9 См см$^{2}$ моль$^{-1}$ при концентрации $6,2\cdot 10^{-3}$ моль/л и 106,1 См см$^{2}$ моль$^{-1}$ при концентрации $1,5\cdot 10^{-2}$ моль/л. Какова удельная электропроводность раствора при бесконечном разбавлении?
\end{Task}
\begin{Task}
Удельная электропроводность насыщенного раствора $AgCl$ в воде при 25$^{o}$C равна $2,28\cdot 10^{-4}$ См м$^{-1}$. а удельная электропроводность воды $1,16\cdot 10^{-4}$ См м$^{-1}$. Рассчитать растворимость $AgCl$ в воде при 25$^{o}$C в моль. л$^{-1}$. 
\end{Task}
\begin{Task}
Удельная электропроводность 4\% водного раствора $H_{2}SO_{4}$ при 18$^{o}$C равна 0,168 См. см$^{-1}$, плотность раствора 1,026 г/см$^{3}$. Рассчитать эквивалентную электропроводность раствора. 
\end{Task}
\begin{Task}
Удельная электропроводность бесконечно разбавленных растворов соляной кислоты, хлорида натрия и ацетата натрия при 25$^{o}$C равна соответственно 425,0, 128,1 и 91,0 См м$^{2}$. моль$^{-1}$. Какова удельная электропроводность бесконечно разбавленного раствора уксусной кислоты при 25$^{o}$C? 
\end{Task}
\begin{Task}
Удельная электропроводность бесконечно разбавленных растворов $KCl$, $KNO_{3}$ и $AgNO_{3}$ при 25$^{o}$C равна соответственно 149,9, 145,0 и 133,4 См м$^{2}$. моль$^{-1}$. Какова удельная электропроводность бесконечно разбавленного раствора AgCl при 25$^{o}$C? 
\end{Task}
\begin{Task}
Рассчитать удельную электропроводность абсолютно чистой воды при 25$^{o}$C. Ионное произведение воды при 25$^{o}$C равно $1,00\cdot 10^{-14}$. 
\end{Task}

