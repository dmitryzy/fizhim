%Термодинамические потенциалы. Критерии протекания самопроизвольных процессов. Второе начало термодинамики. Энтропия.
\chapter{Практическое занятие 2. Термодинамические потенциалы. Критерии протекания самопроизвольных процессов. Второе начало термодинамики. Энтропия.}
\section{Задачи для самостоятельного решения}
\begin{Task}
Рассчитайте изменение энтропии при нагревании 11,2 л азота от 0 до 50~$^{o}$C и одновременном уменьшении давления от 1 атм до 0,01 атм.
\end{Task}
\begin{Task}
Рассчитайте изменение энтропии при образовании 1 м$^{3}$ воздуха из азота и кислорода (20 об.\%) при температуре 25 $^{o}$C и давлении 1 атм.
\end{Task}
\begin{Task}
Рассчитайте изменение энтропии при смешении 5 кг воды при 80~$^{o}$C с 10~кг воды при 20~$^{o}$C. Удельную теплоемкость воды принять равной: $C_{p} = 4,184$ Дж/(г К).
\end{Task}
\begin{Task}
Рассчитайте изменение энтропии при добавлении 200 г льда, находящегося при температуре 0 $^{o}$C, к 200~г воды (90 $^{o}$C) в изолированном сосуде. Теплота плавления льда равна 6,0 кДж/моль.
\end{Task}
\begin{Task}
Рассчитайте изменение энтропии 1000 г метанола в результате его замерзания при -105 $^{o}$C. Теплота плавления твердого метанола при -98~$^{o}$C ($T_{\textrm{пл.}}$) равна 3160~Дж/моль. Теплоемкости твердого и жидкого метанола равны 55,6 и 81,6~Дж/(мольК), соответственно. Объясните, почему энтропия при замерзании уменьшается, хотя процесс -- самопроизвольный.
\end{Task}
\begin{Task}
Пользуясь справочными данными, рассчитайте стандартное изменение энтропии в реакции 
$H_{2}(\textrm{г}) + ЅO_{2}(\textrm{г}) = H_{2}O(\textrm{г})$
при 25 $^{o}$C и при 300 $^{o}$C.
\end{Task}
\begin{Task}
Энергия Гельмгольца одного моля некоторого вещества записывается следующим образом:
$F = a + T(b - c - b\ln T - d \ln V)$,
где $a$, $b$, $c$, $d$ - константы. Найдите давление, энтропию и теплоемкость $C_{v}$ этого тела. Дайте физическую интерпретацию константам $a$, $b$, $d$.
\end{Task}
\begin{Task}
Вычислите изменение $H$, $U$, $F$, $G$, $S$ при одновременном охлаждении от 2000 К до 200 К и расширении от 0,5 м$^{3}$ до 1,35 м$^{3}$ 0,7 молей азота ($C_{v}=\frac{5}{2} R$). Энтропия газа в исходном состоянии равна 150 Дж/(мольК), газ можно считать идеальным.
\end{Task}
\begin{Task}
Вычислите изменение энергии Гиббса при сжатии от 1 атм до 3 атм при 298 К: а) одного моля жидкой воды; б) одного моля водяного пара (идеальный газ).
\end{Task}
\begin{Task}
Вычислите стандартную энергию Гиббса образования ($\Delta_{f}G_{298}^{0}$) жидкой и газообразной воды, если известны следующие данные: 
$\Delta_{f}H_{298}^{0}(H_{2}O(\textrm{г})) = -241,8 $~кДж/моль,  

$\Delta_{f}H_{298}^{0}(H_{2}O(\textrm{ж})) = -285,6$~кДж/моль, 
$S_{298}^{0}(H_{2}) = 130,6$~Дж/(мольК), 

$S_{298}^{0}(O_{2}) = 205,0$~Дж/(мольК),  $S_{298}^{0}(H_{2}O(\textrm{г})) = 188,5$~Дж/(мольК), 

$S_{298}^{0}(H_{2}O(\textrm{ж})) = 69,8$~Дж/(мольК).
\end{Task}
\begin{Task}
Для химической реакции:
$4HCl(\textrm{г}) + O_{2}(\textrm{г}) = 2Cl_{2}(\textrm{г}) + 2H_{2}O(\textrm{ж})$ рассчитайте $\Delta G^{0}$ при 25 $^{o}$C 
Стандартные значения энтальпии образования и абсолютной энтропии при 25$^{o}$C равны: 
$\Delta_{f}H_{298}^{0}(HCl) = -22,1$ ккал/моль, $\Delta_{f}H_{298}^{0}(H_{2}O(\textrm{ж})) = -68,3$ ккал/моль; 

$S^{0}(HCl) = 44,6$ кал/(мольK), $S^{0}(O_{2}) = 49,0$ кал/(мольK), 

$S^{0}(Cl_{2}) = 53,3$ кал/(мольK), $S^{0}(H_{2}O(\textrm{ж})) = 16,7$ кал/(мольK).
\end{Task}
\begin{Task}
Для химической реакции:
$CO_{2}(\textrm{г}) + 4H_{2}(\textrm{г}) = CH_{4}(\textrm{г}) + 2H_{2}O(\textrm{ж})$ рассчитайте $\Delta G_{0}$ при 25~$^{o}$C 
Стандартные значения энтальпии образования и абсолютной энтропии при 25 $^{o}$C равны: 
$\Delta_{f}H_{298}^{0}(CO_{2})=-94,1$ ккал/моль,  $\Delta_{f}H_{298}^{0}(CH_{4}) = -17,9$ ккал/моль,  

$\Delta_{f}H_{298}^{0}(H_{2}O(\textrm{ж})) = -68,3$ ккал/моль; 
$S^{0}(CO_{2}) = 51,1$ кал/(мольK), 

$S^{0}(H_{2}) = 31,2$ кал/(мольK),
$S^{0}(CH_{4}) = 44,5$ кал/(мольK), $S^{0}(H_{2}O(\textrm{ж})) = 16,7$ кал/(мольK).
\end{Task}
\begin{Task}
Рассчитайте стандартные энергии Гиббса и Гельмгольца $\Delta G_{0}$ и $\Delta F_{0}$ при 300~$^{o}$C для химической реакции:
$CO(\textrm{г}) + 3H_{2}(\textrm{г}) = CH_{4}(\textrm{г}) + H_{2}O(\textrm{г}).$
Может ли эта реакция протекать самопроизвольно при данной температуре? 
\end{Task}
\begin{Task}
Рассчитайте стандартные энергии Гиббса и Гельмгольца $\Delta G_{0}$ и $\Delta F_{0}$ при 60~$^{o}$C для химической реакции:
$CH_{3}COOH(\textrm{ж}) + 2H_{2}(\textrm{г}) = C_{2}H_{5}OH(\textrm{ж}) + H_{2}O(\textrm{ж})$.
Может ли эта реакция протекать самопроизвольно при данной температуре? 
\end{Task}
\begin{Task}
Рассчитайте стандартные энергии Гиббса и Гельмгольца $\Delta G_{0}$ и $\Delta F_{0}$ при 700~$^{o}$C для химической реакции:
$CaCO_{3}(\textrm{тв}) = CaO(\textrm{тв}) + CO_{2}(\textrm{г})$. 
Может ли эта реакция протекать самопроизвольно при данной температуре? 
\end{Task}
% Изучение влияния различных факторов на равновесие химической реакции. 
%Указание: во всех задачах считать газы идеальными. 
\begin{Task}
При 1273~К и общем давлении 30~атм в равновесной смеси

$CO_{2}(\textrm{г}) + C(\textrm{тв}) = 2CO(\textrm{г})$ 
содержится 17об.\% $CO_{2}$. Сколько процентов $CO_{2}$ будет содержаться в газе при общем давлении 20~атм? При каком давлении в газе будет содержаться 25 об.\% $CO_{2}$?
\end{Task}
\begin{Task}
При 2273 K и общем давлении 1 атм 2\% воды диссоциировано на водород и кислород. Рассчитать константу равновесия реакции
$H_{2}O(\textrm{г}) = H_{2}(\textrm{г}) + 0,5O_{2}(\textrm{г})$.
\end{Task}
\begin{Task}
Константа равновесия реакции
$CO(\textrm{г}) + H_{2}O(\textrm{г}) = CO_{2}(\textrm{г}) + H_{2}(\textrm{г})$
при 773~K равна $K_{p}=5,5$. Смесь, состоящая из 1~моль $CO$ и 5~моль $H_{2}O$, нагрели до этой температуры. Рассчитать мольную долю $H_{2}O$ в равновесной смеси.
\end{Task}
\begin{Task}
Константа равновесия реакции
$N_{2}O_{4}(\textrm{г}) = 2NO_{2}(\textrm{г})$
при 298~K равна\\ $K_{p} = 0,143$. Рассчитать давление, которое установится в сосуде объемом 1~л, в который поместили 1~г $N_{2}O_{4}$ при этой температуре.
\end{Task}
\begin{Task}
Сосуд объемом 3~л, содержащий $1,79\cdot  10^{-2}$ моль $I_{2}$, нагрели до 973~K. Давление в сосуде при равновесии оказалось равно 0,49~атм. Считая газы идеальными, рассчитать константу равновесия при 973~K для реакции
$I_{2}(\textrm{г}) = 2I (\textrm{г})$.
\end{Task}
\begin{Task}
Для реакции
$PCl_{5}(\textrm{г}) = PCl_{3}(\textrm{г}) + Cl_{2}(\textrm{г})$ 
при 523~K $\Delta_{r}G^{0} = -2508$~Дж/моль. При каком общем давлении степень превращения $PCl_{5}$ при 523~K составит 30\%?
\end{Task}
\begin{Task}
Для реакции
$2HI(\textrm{г}) = H_{2}(\textrm{г}) + I_{2}(\textrm{г})$ 
константа равновесия $K_{p} = 1,83\cdot 10^{-2}$ при 698,6~К. Сколько граммов $HI$ образуется при нагревании до этой температуры 10~г $I_{2}$ и 0,2~г $H_{2}$ в трехлитровом сосуде? Чему равны парциальные давления $H_{2}$, $I_{2}$ и $HI$?
\end{Task}
\begin{Task}
Сосуд объемом 1 л, содержащий 0,341 моль $PCl_{5}$ и 0,233 моль $N_{2}$, нагрели до 523 K. Общее давление в сосуде при равновесии оказалось равно 29,33 атм. Считая все газы идеальными, рассчитать константу равновесия при 523 K для протекающей в сосуде реакции
$PCl_{5}(\textrm{г}) = PCl_{3} (\textrm{г}) + Cl_{2}(\textrm{г})$
\end{Task}
\begin{Task}
Рассчитать общее давление, которое необходимо приложить к смеси 3 частей $H_{2}$ и 1 части $N_{2}$, чтобы получить равновесную смесь, содержащую 10\% $NH_{3}$ по объему при 400~$^{o}$C. Константа равновесия для реакции $N_{2}(\textrm{г}) + 3H_{2}(\textrm{г}) = 2NH_{3}(\textrm{г})$
при 400~$^{o}$C равна $K_{p} = 1,60\cdot 10^{-4}$.
\end{Task}
\begin{Task}
При 250~$^{o}$C и общем давлении 1~атм $PCl_{5}$ диссоциирован на 80\% по реакции \\ $PCl_{5}(\textrm{г}) = PCl_{3}(\textrm{г}) + Cl_{2}(\textrm{г})$. 
Чему будет равна степень диссоциации $PCl_{5}$, если в систему добавить $N_{2}$, чтобы парциальное давление азота было равно 0,9~атм? Общее давление поддерживается равным 1~атм.
\end{Task}
\begin{Task}
При 2273 K для реакции 
$N_{2}(\textrm{г}) + O_{2}(\textrm{г}) = 2NO(\textrm{г})$ 
$K_{p}= 2,5\cdot 10^{-3}$. В равновесной смеси $N_{2}$, $O_{2}$, $NO$ и инертного газа при общем давлении 1~бар содержится 80\% (по объему) $N_{2}$ и 16\% $O_{2}$. Сколько процентов по объему составляет $NO$? Чему равно парциальное давление инертного газа?
\end{Task}