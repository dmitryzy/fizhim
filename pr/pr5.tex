%Растворы
\chapter{ Практическое занятие 5. Растворы}
\section{Задачи для самостоятельного решения}
\begin{Task}
Давления пара чистых $CHCl_{3}$ и $CCl_{4}$ при 25$^{o}$ C равны 26,54 и 15,27 кПа. Полагая, что они образуют идеальный раствор, рассчитать давление пара и состав (в мольных долях) пара над раствором, состоящим из 1 моль $CHCl_{3}$ и 1 моль $CCl_{4}$.
\end{Task}
\begin{Task}
Дибромэтилен и дибромпропилен при смешении образуют почти идеальные растворы. При 80$^{o}$C давление пара дибромэтилена равно 22,9 кПа, а дибромпропилена 16,9 кПа. Рассчитать состав пара, находящегося в равновесии с раствором, мольная доля дибромэтилена в котором равна 0,75. Рассчитать состав раствора, находящегося в равновесии с паром, мольная доля дибромэтилена в котором равна 0,50.
\end{Task}
\begin{Task}
Этанол и метанол при смешении образуют почти идеальные растворы. При 20$^{o}$C давление пара этанола равно 5,93 кПа, а метанола 11,83 кПа. Рассчитать давление пара раствора, состоящего из 100 г этанола и 100 г метанола, а также состав (в мольных долях) пара над этим раствором при 20$^{o}$C.
\end{Task}
\begin{Task}
Давления пара чистых бензола и толуола при 60$^{o}$C равны 51,3 и 18,5 кПа. При каком давлении закипит при 60$^{o}$C раствор, состоящий из 1 моля бензола и 2 молей толуола? Каков будет состав пара?
\end{Task}
\begin{Task}
Давления пара чистых $C_{6}H_{5}Cl$ и $C_{6}H_{5}Br$ при 140$^{o}$C равны 1,237 бар и 0,658 бар. Рассчитать состав раствора $C_{6}H_{5}Cl$ --- $C_{6}H_{5}Br$, который при давлении 1 бар кипит при температуре 140$^{o}$C, а также состав образующегося пара. Каково будет давление пара над раствором, полученным конденсацией образующегося пара?
\end{Task}
\begin{Task}
Константа Генри для $CO_{2}$ в воде при 25$^{o}$C равна 1,25  106 Торр. Рассчитать растворимость (в единицах моляльности) $CO_{2}$ в воде при 25$^{o}$C, если парциальное давление $CO_{2}$ над водой равно 0,1 атм.
\end{Task}
\begin{Task}
Константы Генри для кислорода и азота в воде при 25$^{o}$C равны $4,40\cdot 10^{9}$ Па и $8,68\cdot 10^{9}$ Па соответственно. Рассчитать состав (в \%) воздуха, растворенного в воде при 25$^{o}$C, если воздух над водой состоит из 80\% $N_{2}$ и 20\% $O_{2}$ по объему, а его давление равно 1 бар.
\end{Task}
\begin{Task}
Константы Генри для кислорода и азота в воде при 0$^{o}$C равны $2,54\cdot 10^{4}$ бар и $5,45\cdot 10^{4}$ бар соответственно. Рассчитать понижение температуры замерзания воды, вызванное растворением воздуха, состоящего из 80\% $N_{2}$ и 20\% $O_{2}$ по объему при давлении 1,0 бар. Криоскопическая константа воды равна 1,86 К кг/моль.
\end{Task}
\begin{Task}
При 25$^{o}$C давление пара хлорметана над его раствором в углеводороде при разных мольных долях следующее:\\
\begin{tabular}{|c|c|c|c|c|}
\hline 
$X_{CH_{3}Cl}$ (р-р) & 0,005 & 0,009 & 0,019 7 & 0,024\\
\hline
$P_{CH_{3}Cl}$, Торр & 205 & 363 & 756 & 946\\
\hline
\end{tabular} 

Показать, что в этом интервале мольных долей раствор подчиняется закону Генри и рассчитать константу Генри.
\end{Task}
\begin{Task}
При 57,2$^{o}$C и давлении 1,00 атм мольная доля ацетона в паре над раствором ацетон-метанол с мольной долей ацетона в растворе $x_{A} = 0,400$ равна $y_{A} = 0,516$. Рассчитать активности и коэффициенты активности обоих компонентов в этом растворе на основе закона Рауля. Давления пара чистых ацетона и метанола при этой температуре равны 786 и 551 Торр соответственно.
\end{Task}
\begin{Task}
Для раствора этанол -- хлороформ при 35$^{o}$C получены следующие данные:\\
\begin{tabular}{|c|c|c|c|c|c|c|}
\hline 
$x$ этанола (р-р) & 0 & 0,2 & 0,4 & 0,6 & 0,8 & 1,0\\
\hline 
$y$ этанола (пар) & 0 & 0,1382 & 0,1864 & 0,2554 & 0,4246 & 1,0000\\
\hline 
$P$ общее, кПа & 39,345 & 40,559 & 38,690 & 34,387 & 25,357 & 13,703\\
\hline 
\end{tabular} 

Рассчитать коэффициенты активности обоих компонентов в растворе на основе закона Рауля.
\end{Task}
\begin{Task}
Для раствора $CS_{2}$ --- ацетон при 35,2$^{o}$C получены следующие данные:\\
\begin{tabular}{|c|c|c|c|c|c|c|}
\hline 
$x_{CS_{2}}$ (р-р) & 0 & 0,2 & 0,4 & 0,6 & 0,8 & 1,0\\
\hline 
$P_{CS_{2}}$, кПа & 0 & 37,3 & 50,4 & 56,7 & 61,3 & 68,3\\
\hline 
$P$ ацетона, кПа & 45,9 & 38,7 & 34,0 & 30,7 & 25,3 & 0\\
\hline 
\end{tabular} 

Рассчитать коэффициенты активности обоих компонентов в растворе на основе закона Рауля.
\end{Task}
\begin{Task}
Для раствора вода --- н-пропанол при 25$^{o}$C получены следующие данные:\\
\begin{tabular}{|c|c|c|c|c|c|c|c|c|c|}
\hline 
$x$ н-пропанола (р-р) & 0 & 0,02 & 0,05 & 0,10 & 0,20 & 0,40 & 0,60 & 0,80 & 1,00\\
\hline  
$p$ воды, кПа & 3,17 & 3,13 & 3,09 & 3,03 & 2,91 & 2,89 & 2,65 & 1,79 & 0,00\\ 
\hline 
$p$ н-пропанола, кПа & 0,00 & 0,67 & 1,44 & 1,76 & 1,81 & 1,89 & 2,07 & 2,37 & 2,90\\
\hline 
\end{tabular} 

Рассчитать активности и коэффициенты активности обоих компонентов в растворе с мольной долей н-пропанола 0,20, 0,40, 0,60 и 0,80 на основе законов Рауля и Генри, считая воду растворителем.
\end{Task}
\begin{Task}
Парциальные мольные объемы воды и метанола в растворе с мольной долей метанола 0,4 равны 17,35 и 39,01 см$^{3}$/моль соответственно. Рассчитать объем раствора, содержащего 0,4 моль метанола и 0,6 моль воды, а также объем до смешения. Плотности воды и метанола равны 0,998 и 0,791 г/см$^{3}$ соответственно.
\end{Task}
\begin{Task}
Парциальные мольные объемы воды и этанола в растворе с мольной долей этанола 0,2 равны 17,9 и 55,0 см$^{3}$/моль соответственно. Рассчитать объемы воды и этанола, необходимые для приготовления 1 л такого раствора. Плотности воды и этанола равны 0,998 и 0,789 г/см$^{3}$ соответственно.
\end{Task}
\begin{Task}
Парциальные мольные объемы ацетона и хлороформа в растворе с мольной долей хлороформа 0,4693 равны 74,166 и 80,235 см$^{3}$/моль соответственно. Рассчитать объем такого раствора, имеющего массу 1 кг.
\end{Task}
\begin{Task}
Плотность 50\% (по массе) раствора этанола в воде при 25$^{o}$C равна 0,914 г/см$^{3}$. Рассчитать парциальный мольный объем этанола в этом растворе, если парциальный мольный объем воды равен 17,4 см$^{3}$/моль.
\end{Task}
\begin{Task}
Общий объем раствора этанола ($V$, мл), содержащего 1,000 кг воды, при 25$^{o}$C описывается выражением
$$V = 1002,93 + 54,6664\cdot m-0,36394\cdot m^{2} + 0,028256\cdot m^{3}$$
где $m$ -- моляльность раствора. Рассчитать парциальные мольные объемы воды и этанола в растворе, состоящем из 1,000 кг воды и 0,500 кг этанола.
\end{Task}
\begin{Task}
Парциальный мольный объем ($V$, см$^{3}$/моль) $K_{2}SO_{4}$ в водном растворе при 25$^{o}$C описывается выражением
$$V = 32,28 + 18,216\cdot m^{0,5}$$
где $m$ -- моляльность раствора. Используя уравнение Гиббса-Дюгема, получите выражение для парциального мольного объема воды в этом растворе. Мольный объем чистой воды при 25$^{o}$C равен 18,079 см$^{3}$/моль.
\end{Task}